\documentclass[10pt]{beamer}


\input{includes}
\title{Langage Python}
\subtitle{M2 Ingénierie Mathématique, Sorbonne Université}
\author{Baptiste Gregorutti\\Ingénieur de Recherche}
\date{Novembre 2024}
\titlegraphic{\hfill\includegraphics[height=1.5cm]{img/logo_sorbonne.png}}


\begin{document}

\maketitle

\begin{frame}{}
  \centering
  \Large
  \textbf{Qualité de code}
\end{frame}


\section{Structure d'un programme}

\begin{frame}[fragile]{Structurer un programme, version simple}

  \begin{block}{Structure en plusieurs fichiers}
    \begin{itemize}
      \item Un ou plusieurs modules contenant les fonctionnalités implémentées
      \item Un programme principal
    \end{itemize}
  \end{block}

\bigskip

\begin{lstlisting}[numbers=none]
my_project/
  - module1.py
  - module2.py
  - main.py
\end{lstlisting}
\end{frame}

\begin{frame}[fragile]{Structurer un programme}

  \texttt{module1.py}
\begin{lstlisting}[language=python, numbers=none]
import math
import sys

def prod(a, b):
    return a * b

def add(a, b):
    return a + b
\end{lstlisting} 

\texttt{main.py}

\begin{lstlisting}[language=python, numbers=none]
from module1 import add, prod
print(add(1, 2))
print(prod(1, 2))
\end{lstlisting} 
\end{frame}


\begin{frame}[fragile]{Structurer un programme}

  \begin{block}{Contrainte}
    \medskip

    Si les fichiers \texttt{main.py} et \texttt{module1.py} sont dans le même dossier, vous pouvez faire
\begin{lstlisting}[language=bash, numbers=none]
python main.py
\end{lstlisting}

  Sinon, ça ne fonctionnera pas !
  \end{block}
\end{frame}


\section{Conventions de code à retenir}

\begin{frame}{PEP8}

  \begin{block}{PEP8 -- Style de code}
    \medskip
    Les Python Enhancement Proposals, ou PEP, sont un ensemble de recommandations liées au langage python.

    La PEP8 définit des conventions \textbf{assez restrictives} de style de code.
  \end{block}

  Page officielle : \url{peps.python.org/pep-0008}

\end{frame}


\begin{frame}[fragile]{Longueur des lignes}
  \begin{block}{}
    \medskip
    La PEP8 recommande entre 80 et 100 charactères maximum.

    Je recommande \textbf{150 charactères}.
  \end{block}
\end{frame}


\begin{frame}[fragile]{Indentation}
  \begin{block}{}
    \medskip
    Utiliser une identation à \textbf{4 espaces exactement} et ne pas utiliser de \textbf{tabulation}.
    
    Exemple pour un appel de fonction
\begin{lstlisting}[language=python, numbers=none]
foo = long_function_name(var_one, var_two,
                         var_three, var_four)
\end{lstlisting}

ou bien

\begin{lstlisting}[language=python, numbers=none]
foo = long_function_name(
    var_one, var_two,
    var_three, var_four)
\end{lstlisting}

    Exemple pour la définition d'une liste
\begin{lstlisting}[language=python, numbers=none]
l = [[1, 2, 3,
      4, 5, 6]]
\end{lstlisting}
      
  \end{block}
\end{frame}



\begin{frame}[fragile]{Blocs de code et retours à la ligne}
  \begin{block}{}
    \medskip

    \begin{enumerate}
      \item Coder des blocs de code \textbf{cohérents}
      \item Séparer les blocs de code par un \textbf{unique} retour à la ligne.
    \end{enumerate}
  \end{block}
\end{frame}




\begin{frame}[fragile]{Encoding et langue}

  \begin{block}{}
    \medskip
    Utiliser l'anglais et proscrire les caractères spéciaux (encodage UTF-8)
  \end{block}
\end{frame}


\begin{frame}[fragile]{Imports}
  \begin{block}{}
    \medskip
    \begin{itemize}
      \item Un import par ligne
      \item Préférer les imports \textbf{absolus}
      \item Ordre des imports : package de la librairie standard, librairies installées puis librairies locales

Par exemple :
\begin{lstlisting}[language=python, numbers=none]
import os
import sys
import time

import numpy as np

import mypackage
\end{lstlisting}
    \end{itemize}
  \end{block}
\end{frame}



\begin{frame}[fragile]{Chaîne de charactères}
  \begin{block}{}
    \medskip
    Il est possible de définir des chaînes de caractères avec " " ou bien ' '.

    PEP8 ne donne pas de recommandation mais il est important de ne pas utiliser les deux façons de faire (uniformisation du code)
  \end{block}
\end{frame}



\begin{frame}[fragile]{``Whitespaces'' dans les expressions}
  \begin{block}{}
    \medskip
    Proscrire les `espaces blancs' inutiles dans les situations suivantes
  \end{block}

  Correct :
\begin{lstlisting}[language=python, numbers=none]
spam(ham[1], {eggs: 2})
foo = (0,)
if x == 4: print(x, y); x, y = y, x
spam(1)
dct['key'] = lst[index]
\end{lstlisting}

Incorrect :
\begin{lstlisting}[language=python, numbers=none]
spam( ham[ 1 ], { eggs: 2 } )
bar = (0, )
if x == 4 : print(x , y) ; x , y = y , x
spam (1)
dct ['key'] = lst [index]
\end{lstlisting}
\end{frame}

\begin{frame}[fragile]{``Whitespaces'' dans les expressions (suite)}
  \begin{block}{}
    \medskip
    Proscrire les `espaces blancs' inutiles dans les situations suivantes
  \end{block}

  Correct :
\begin{lstlisting}[language=python, numbers=none]
x = 1
y = 2
long_variable = 3
\end{lstlisting}

Incorrect :
\begin{lstlisting}[language=python, numbers=none]
x             = 1
y             = 2
long_variable = 3
\end{lstlisting}
\end{frame}


\begin{frame}[fragile]{``Whitespaces'' dans les expressions (suite)}
  \begin{block}{}
    \medskip
    Proscrire les `espaces blancs' inutiles dans les situations suivantes
  \end{block}

  Correct :
\begin{lstlisting}[language=python, numbers=none]
def complex(real, imag=0.0):
    return magic(r=real, i=imag)
\end{lstlisting}

Incorrect :
\begin{lstlisting}[language=python, numbers=none]
def complex(real, imag = 0.0):
    return magic(r = real, i = imag)
\end{lstlisting}
\end{frame}




\begin{frame}[fragile]{``Whitespaces'' dans les expressions (suite)}
  \begin{block}{}
    \medskip
    MAIS ne pas supprimer tous les espaces blancs, par souci de lisibilité !
  \end{block}

  Correct :
\begin{lstlisting}[language=python, numbers=none]
i = i + 1
submitted += 1
x = x*2 - 1
hypot2 = x*x + y*y
c = (a+b) * (a-b)
\end{lstlisting}

Incorrect :
\begin{lstlisting}[language=python, numbers=none]
i=i+1
submitted +=1
x = x * 2 - 1
hypot2 = x * x + y * y
c = (a + b) * (a - b)
\end{lstlisting}
\end{frame}




\begin{frame}[fragile]{Docstrings}
  \begin{block}{}
    \medskip
    Les docstrings sont des commentaires insérés dans le code, en début de modules ou de fonctions.
    Ils résument ce que fait la fonction et indiquent quels sont les arguments et les valeurs de retour.
  \end{block}

\begin{lstlisting}[language=python, numbers=none]
def sample(n):
    """
    Get a sample from Gaussian distribution

    Args:
        n: the number of desired normal values

    Returns
        A sequence of n Gaussian values
    """
    return normal(0, 1, number)
\end{lstlisting}

  Voir aussi la PEP257 : \url{peps.python.org/pep-0257}
\end{frame}

\begin{frame}{Conventions de nommage}
  \begin{block}{}
    \medskip
    Il existe différentes façon de nommer en python :
    \begin{itemize}
      \item \texttt{lowercase} ou \texttt{snake\_case}
      \item \texttt{lowercase\_with\_underscores}
      \item \texttt{UPPERCASE}
      \item \texttt{UPPERCASE\_WITH\_UNDERSCORES}
      \item \texttt{CapitalizedWords}
    \end{itemize}
  \end{block}
\end{frame}


\begin{frame}{Conventions de nommage}
  \begin{block}{}
    \medskip
    Les conventions de nommage sont les suivantes :
    \begin{itemize}
      \item Packages : court, \texttt{lowercase}
      \item Modules : court, \texttt{lowercase} ou \texttt{lowercase\_with\_underscores}
      \item Classes : \texttt{CapitalizedWords}
      \item Variables locales : \texttt{lowercase} ou \texttt{lowercase\_with\_underscores}
      \item Variables globales : \texttt{UPPERCASE} ou \texttt{UPPERCASE\_WITH\_UNDERSCORES}
    \end{itemize}

  \end{block}
\end{frame}

\section*{Package \texttt{pylint}}

\begin{frame}[fragile]{Package \texttt{pylint}}
  \begin{block}{Analyse de code}
    \medskip
    \texttt{pylint} est un outil permettant de mesurer la qualité de code selon les conventions de la PEP8 et met une note sur 10
  \end{block}

  A la racine de votre package
\begin{lstlisting}[numbers=none]
python -m pylint mypackage
\end{lstlisting}

  ou simplement
\begin{lstlisting}[numbers=none]
python -m pylint mymodule.py
\end{lstlisting}
    
\end{frame}



\begin{frame}[fragile]{Package \texttt{pylint}}
  \begin{block}{Exemple}
    \medskip
      \only<1>{
        \texttt{python -m pylint sequence\_adn.py}
        \begin{center}
          \includegraphics[width=\textwidth]{img/pylint1.png}
        \end{center}
      }

      \only<2>{
        \texttt{python -m pylint sequence\_adn.py --max-line-length=150}
        \begin{center}
          \includegraphics[width=\textwidth]{img/pylint2.png}
        \end{center}
      }
  \end{block}
    
\end{frame}


\begin{frame}[fragile]{A partir de maintenant}

  \begin{enumerate}
    \item Commencer à implémenter dans un fichier \texttt{.py}
    \item Appliquer les conventions PEP8
    \item Analyser votre code avec \texttt{pylint}
\begin{lstlisting}[numbers=none]
python -m pylint module.py --max-line-length=150
\end{lstlisting}
  \end{enumerate}
  
\end{frame}

\end{document}