\documentclass[10pt]{beamer}


\input{includes}
\title{Langage Python}
\subtitle{M2 Ingénierie Mathématique, Sorbonne Université}
\author{Baptiste Gregorutti\\Ingénieur de Recherche}
\date{Novembre 2024}
\titlegraphic{\hfill\includegraphics[height=1.5cm]{img/logo_sorbonne.png}}


\begin{document}

\maketitle

\begin{frame}{}
  \centering
  \Large
  \textbf{Partie 4}

  \textbf{Packaging, tests unitaires et qualité de code}
\end{frame}


\section{Rappels}

\begin{frame}[fragile]{Structurer un programme, version simple}

  \begin{block}{Structure en plusieurs fichiers}
    \begin{itemize}
      \item Un ou plusieurs modules contenant les fonctionnalités implémentées
      \item Un programme principal
    \end{itemize}
  \end{block}

\bigskip

\begin{lstlisting}[numbers=none]
my_project/
  - module1.py
  - module2.py
  - main.py
\end{lstlisting}
\end{frame}


\begin{frame}[fragile]{Structurer un programme, version simple}

  \underline{Cas 1} : un seul fichier avec l'instruction \texttt{if \_\_name\_\_ == "\_\_main\_\_"}

\begin{lstlisting}[language=python, numbers=none]
import math
import sys

def prod(a, b):
    return a * b

def add(a, b):
    return a + b

if __name__ == "__main__":
    print(add(1, 2))
    print(prod(1, 2))
\end{lstlisting} 

\end{frame}


\begin{frame}[fragile]{Structurer un programme, version simple}

  \underline{Cas 2} : deux fichiers

  \texttt{module1.py}
\begin{lstlisting}[language=python, numbers=none]
import math
import sys

def prod(a, b):
    return a * b

def add(a, b):
    return a + b
\end{lstlisting} 

\texttt{main.py}

\begin{lstlisting}[language=python, numbers=none]
from module1 import add, prod
print(add(1, 2))
print(prod(1, 2))
\end{lstlisting} 
\end{frame}


\begin{frame}[fragile]{Structurer un programme, version simple}

  \begin{block}{Problème !}
    \medskip

    Si les fichiers \texttt{main.py} et \texttt{module1.py} sont dans le même dossier, vous pouvez faire
\begin{lstlisting}[language=bash, numbers=none]
python main.py
\end{lstlisting}

  Sinon, ça ne fonctionnera pas !
  \end{block}
\end{frame}



\section{Packaging}

\begin{frame}[fragile]{Qu'est ce qu'un package ?}

  \begin{block}{Un package python est }
    \medskip
    \begin{enumerate}
      \item un \textcolor{red}{ensemble de modules} organisés en dossiers et sous-dossiers
      \item un moyen de créer des libraries spécifiques à un \textcolor{red}{domaine particulier}
      \item \textcolor{red}{installable} via pip ou conda
      \item un moyen d'utiliser les fonctionnalités dans \textcolor{red}{un autre package}
    \end{enumerate}

    Par exemple :
    \begin{itemize}
      \item SciPy : dédié au calcul scientifique
      \item NumPy : dédié au calcul numérique et matriciel
      \item Pandas : manipulation de tableaux de données et analyses statistiques
      \item Matplotlib : dédié à la visualisation de données
    \end{itemize}
  \end{block}
  

\end{frame}


\begin{frame}[fragile]{Exemple}

\begin{lstlisting}[numbers=none]
package_folder/
  - README.md	
  - setup.cfg
  - setup.py
  - mypackage/
    - __init__.py
    - mymodule.py
    - mymodule2.py
  - test/
    - test_mymodule.py
    - test_mymodule2.py
\end{lstlisting}

Installation de la librairie
\begin{lstlisting}[numbers=none]
pip install mypackage
\end{lstlisting}

Utilisation :
\begin{lstlisting}[language=python, numbers=none]
>>> from mypackage import mymodule
>>> mymodule.sum(1, 1)
2
\end{lstlisting}
\end{frame}

\begin{frame}[fragile]{Fichier \texttt{\_\_init\_\_.py}}
  
  \begin{itemize}
    \item Fichier nécessaire pour importer le dossier en tant que package
    \item Il est en général vide mais il peut contenir des instructions comme par exemple la version du package
  \end{itemize}
\begin{lstlisting}[language=python, numbers=none]
__version__ = "0.0.0"
\end{lstlisting}
\end{frame}


\begin{frame}[fragile]{Fichiers \texttt{setup.py} et \texttt{setup.cfg}}
  Les fichiers \texttt{setup.py} et \texttt{setup.cfg} permettent de configurer et d'installer le package

  \begin{block}{\texttt{setup.py}}
    \medskip
\begin{lstlisting}[language=python, numbers=none]
from setuptools import setup
setup()
\end{lstlisting}
  \end{block}

  \begin{block}{\texttt{setup.cfg}}
    \medskip
\begin{lstlisting}[numbers=none]
[metadata]
name = mypackage
version = attr: mypackage.__version__
description = My package description
long_description = file: README.md

[options]
packages = find:
install_requires =
  numpy
\end{lstlisting}
    
  \end{block}
\end{frame}

\begin{frame}{Fichier \texttt{README.md}}
  Le fichier \texttt{README.md} est un fichier de documentation, écrit en langage ``Markdown'', comme dans les Notebooks Jupyter.

  Il doit décrire précisément les fonctionnalités implémentées dans le packages
\end{frame}

\begin{frame}[fragile]{Les commandes à connaitre}

  Installer le package localement par l'exécution du programme \texttt{setup.py}
\begin{lstlisting}[numbers=none]
python setup.py install
\end{lstlisting}

Installer le package localement avec \texttt{pip}
\begin{lstlisting}[numbers=none]
pip install .
\end{lstlisting}

Compiler le package dans une archive tar.gz
\begin{lstlisting}[numbers=none]
python setup.py sdist
\end{lstlisting}

Installer le package depuis l'archive tar.gz
\begin{lstlisting}[numbers=none]
pip install dist/mypackage-0.0.0.tar.gz
\end{lstlisting}

\end{frame}


\begin{frame}{A vous de jouer}

  \begin{enumerate}
    \item Créez un package \texttt{mypackage} avec les fonctions implémentée dans le TP ``sequence\_adn''
    \item Créez le fichier \texttt{\_\_init\_\_.py} contenant la version de votre package
    \item Créez les fichiers \texttt{setup.py} et \texttt{setup.cfg}
    \item Dans un terminal / invite de commanda Anaconda
    \begin{itemize}
      \item Placez-vous dans le dossier contenant le package et l'installez
      \item Placez-vous dans un autre dossier et testez en important le package installé
    \end{itemize}
  \end{enumerate}
  
\end{frame}



\section{Tests unitaires}

\begin{frame}[fragile]{Tests unitaires}
  
  \onslide<1->
  \begin{block}{Définition}
    \medskip    
    Un test unitaire est une routine permettant de vérifier le bon fonctionnement d'une fonction du package
  \end{block}

  \onslide<2->
  \begin{exampleblock}{Par exemple}
\begin{lstlisting}[language=python, numbers=none]
def generate(number):
  return np.random.normal(0, 1, number)
\end{lstlisting}
  \end{exampleblock}
  
  \onslide<3->
  Comment tester la function \texttt{generate} ?
  \begin{itemize}
    \item Est-ce que \texttt{generate(10)} est un \texttt{np.array} ?
    \item Est-ce que \texttt{generate(10)} donne la bonne distribution ?
    \item Est-ce que \texttt{len(generate(10))} est effectivement 10 ?
  \end{itemize}
\end{frame}


\begin{frame}[fragile]{Implémenter des tests unitaires}

  \begin{block}{Les clés pour implémenter des tests}
    \medskip

    \begin{enumerate}
      \item<+-> Les fonctions de test sont placées dans un dossier \texttt{test/}
      \item<+->Les fonctions de test utilisent le mot clé ``assert''
\begin{lstlisting}[language=python, numbers=none, morekeywords=assert]
from mypackage.mymodule import generate
def test_generate():
    assert len(generate(10)) == 10
\end{lstlisting}
      \item<+-> Conventions pour les noms des fonctions de test
        \begin{itemize}
          \item<+-> Un \textcolor{blue}{module de test} par module de package.
          
          Ex : pour \textcolor{blue}{tester le module} \texttt{mymodule.py}, le nom du fichier de test sera \texttt{test\_mymodule.py}
          \item<+-> Une \textcolor{red}{fonction de test} par fonction testée. 
          
          Ex : pour \textcolor{red}{tester la fonction} \texttt{generate}, le nom de la fonction de test sera \texttt{test\_generate}
        \end{itemize}
    \end{enumerate}
  \end{block}

\end{frame}

\begin{frame}[fragile]{Exemple}

  Pour tester la fonction
  \begin{overprint}

    \onslide<1>
\begin{lstlisting}[language=python, numbers=none]
def hello(name):
    return f"Hello {name}"
\end{lstlisting}

le test est 
\begin{lstlisting}[language=python, numbers=none]
def test_hello():
    assert hello("python") == "Hello python"
\end{lstlisting}

    \onslide<2>
\begin{lstlisting}[language=python, numbers=none]
def hello(name):
    return f"Hello name"
\end{lstlisting}

le test est 
\begin{lstlisting}[language=python, numbers=none]
def test_hello():
    assert hello("python") == "Hello python"
\end{lstlisting}

\begin{center}
  \includegraphics[width=\textwidth]{img/pytest_error.png}
\end{center}


\end{overprint}

\end{frame}


\begin{frame}{Quelques conseils}
  \begin{block}{Il est important}
    \medskip
    \begin{itemize}
      \item<+-> d'implémenter des modules python \textbf{par thème}
      \item<+-> d'implémenter une fonction par ``fonctionnalité'' et éviter du code trop lourd
      \item<+-> de proscrire la \textbf{duplication de code}
      \item<+-> de tester chaque fonctionnalité en un \textbf{minimum de code}
    \end{itemize}
  \end{block}
\end{frame}


\begin{frame}[fragile]{Outils de tests unitaires : Pytest}
  
  \begin{block}{Deux packages à connaitre}
    \medskip
    \begin{itemize}
      \item<+-> \texttt{pytest} : à la racine du projet, exécuter dans un terminal
\begin{lstlisting}[language=python, numbers=none]
python -m pytest -v
\end{lstlisting}
      \item<+-> \texttt{pytest-cov} : un plugin pour mesurer la couverture des tests
\begin{lstlisting}[language=python, numbers=none]
python -m pytest -v --cov
\end{lstlisting}

  \end{itemize}
  \end{block}
  
  \onslide<3>{
    Le package \texttt{unittest} est également utilisé pour implémenter des tests unitaires avec une syntaxe différente de \texttt{pytest}.
  }
\end{frame}

\begin{frame}{A vous de jouer}

  \begin{block}{Dans votre package}
    \medskip
    \begin{itemize}
      \item Implémentez des tests unitaires pour \textbf{chacune des fonctions}
      \item Exécutez les tests avec \texttt{pytest}
    \end{itemize}
  \end{block}
  
\end{frame}




\section{Qualité de code}

\begin{frame}{PEP8}

  \begin{block}{PEP8 -- Style de code}
    \medskip
    Les Python Enhancement Proposals, ou PEP, sont un ensemble de recommandations liées au langage python.

    La PEP8 définit des conventions \textbf{assez restrictives} de style de code.
  \end{block}

  Page officielle : \url{peps.python.org/pep-0008}

\end{frame}


\begin{frame}[fragile]{Conventions PEP8 à retenir}
  \begin{block}{Longueur des lignes}
    \medskip
    La PEP8 recommande entre 80 et 100 charactères maximum.

    Je recommande \textbf{150 charactères}.
  \end{block}
\end{frame}


\begin{frame}[fragile]{Conventions PEP8 à retenir}
  \begin{block}{Indentation}
    \medskip
    Utiliser une identation à \textbf{4 espaces exactement} et ne pas utiliser de \textbf{tabulation}.
    
    Exemple pour un appel de fonction
\begin{lstlisting}[language=python, numbers=none]
foo = long_function_name(var_one, var_two,
                         var_three, var_four)
\end{lstlisting}

ou bien

\begin{lstlisting}[language=python, numbers=none]
foo = long_function_name(
    var_one, var_two,
    var_three, var_four)
\end{lstlisting}

    Exemple pour la définition d'une liste
\begin{lstlisting}[language=python, numbers=none]
l = [1, 2, 3,
     4, 5, 6]]
\end{lstlisting}
      
  \end{block}
\end{frame}



\begin{frame}[fragile]{Conventions PEP8 à retenir}
  \begin{block}{Blocs de code et retours à la ligne}
    \medskip
    Coder des blocs de code \textbf{cohérents}
    
    Séparés les blocs de code par un \textbf{unique} retour à la ligne.
  \end{block}

  \begin{block}{Encoding et langue}
    \medskip
    Utiliser l'anglais et proscrire les caractères spéciaux (encodage UTF-8)
  \end{block}
\end{frame}


\begin{frame}[fragile]{Conventions PEP8 à retenir}
  \begin{block}{Imports}
    \medskip
    \begin{itemize}
      \item Un import par ligne
      \item Préférer les imports \textbf{absolus}
      \item Ordre des imports : package de la librairie standard, librairies installées puis librairies locales

Par exemple :
\begin{lstlisting}[language=python, numbers=none]
import os
import sys
import time

import numpy as np

import mypackage
\end{lstlisting}
    \end{itemize}
  \end{block}
\end{frame}



\begin{frame}[fragile]{Conventions PEP8 à retenir}
  \begin{block}{Chaîne de charactères}
    \medskip
    Il est possible de définir des chaînes de caractères avec " " ou bien ' '.

    PEP8 ne donne pas de recommandation mais il est important de ne pas utiliser les deux façons de faire (uniformisation du code)
  \end{block}
\end{frame}



\begin{frame}[fragile]{Conventions PEP8 à retenir}
  \begin{block}{``Whitespaces'' dans les expressions}
    \medskip
    Proscrire les `espaces blancs' inutiles dans les situations suivantes
  \end{block}

  Correct :
\begin{lstlisting}[language=python, numbers=none]
spam(ham[1], {eggs: 2})
foo = (0,)
if x == 4: print(x, y); x, y = y, x
spam(1)
dct['key'] = lst[index]
\end{lstlisting}

Incorrect :
\begin{lstlisting}[language=python, numbers=none]
spam( ham[ 1 ], { eggs: 2 } )
bar = (0, )
if x == 4 : print(x , y) ; x , y = y , x
spam (1)
dct ['key'] = lst [index]
\end{lstlisting}
\end{frame}

\begin{frame}[fragile]{Conventions PEP8 à retenir}
  \begin{block}{``Whitespaces'' dans les expressions (suite)}
    \medskip
    Proscrire les `espaces blancs' inutiles dans les situations suivantes
  \end{block}

  Correct :
\begin{lstlisting}[language=python, numbers=none]
x = 1
y = 2
long_variable = 3
\end{lstlisting}

Incorrect :
\begin{lstlisting}[language=python, numbers=none]
x             = 1
y             = 2
long_variable = 3
\end{lstlisting}
\end{frame}


\begin{frame}[fragile]{Conventions PEP8 à retenir}
  \begin{block}{``Whitespaces'' dans les expressions (suite)}
    \medskip
    Proscrire les `espaces blancs' inutiles dans les situations suivantes
  \end{block}

  Correct :
\begin{lstlisting}[language=python, numbers=none]
def complex(real, imag=0.0):
    return magic(r=real, i=imag)
\end{lstlisting}

Incorrect :
\begin{lstlisting}[language=python, numbers=none]
def complex(real, imag = 0.0):
    return magic(r = real, i = imag)
\end{lstlisting}
\end{frame}




\begin{frame}[fragile]{Conventions PEP8 à retenir}
  \begin{block}{``Whitespaces'' dans les expressions (suite)}
    \medskip
    MAIS ne pas supprimer tous les espaces blancs, par souci de lisibilité !
  \end{block}

  Correct :
\begin{lstlisting}[language=python, numbers=none]
i = i + 1
submitted += 1
x = x*2 - 1
hypot2 = x*x + y*y
c = (a+b) * (a-b)
\end{lstlisting}

Incorrect :
\begin{lstlisting}[language=python, numbers=none]
i=i+1
submitted +=1
x = x * 2 - 1
hypot2 = x * x + y * y
c = (a + b) * (a - b)
\end{lstlisting}
\end{frame}




\begin{frame}[fragile]{Conventions PEP8 à retenir}
  \begin{block}{Docstrings}
    \medskip
    Les docstrings sont des commentaires insérés dans le code, en début de modules ou de fonctions.
    Ils résument ce que fait la fonction et indiquent quels sont les arguments et les valeurs de retour.
  \end{block}

\begin{lstlisting}[language=python, numbers=none]
def sample(n):
    """
    Get a sample from Gaussian distribution

    Args:
        n: the number of desired normal values

    Returns
        A sequence of n Gaussian values
    """
    return normal(0, 1, number)
\end{lstlisting}

  Voir aussi la PEP257 : \url{peps.python.org/pep-0257}
\end{frame}

\begin{frame}{Conventions PEP8 à retenir}
  \begin{block}{Conventions de nommage}
    \medskip
    Il existe différentes façon de nommer en python :
    \begin{itemize}
      \item \texttt{lowercase} ou \texttt{snake\_case}
      \item \texttt{lowercase\_with\_underscores}
      \item \texttt{UPPERCASE}
      \item \texttt{UPPERCASE\_WITH\_UNDERSCORES}
      \item \texttt{CapitalizedWords}
    \end{itemize}
  \end{block}
\end{frame}


\begin{frame}{Conventions PEP8 à retenir}
  \begin{block}{Conventions de nommage}
    \medskip
    Les conventions de nommage sont les suivantes :
    \begin{itemize}
      \item Packages : court, \texttt{lowercase}
      \item Modules : court, \texttt{lowercase} ou \texttt{lowercase\_with\_underscores}
      \item Classes : \texttt{CapitalizedWords}
      \item Variables locales : \texttt{lowercase} ou \texttt{lowercase\_with\_underscores}
      \item Variables globales : \texttt{UPPERCASE} ou \texttt{UPPERCASE\_WITH\_UNDERSCORES}
    \end{itemize}

  \end{block}
\end{frame}



\begin{frame}[fragile]{Package \texttt{pylint}}
  \begin{block}{Analyse de code}
    \medskip
    \texttt{pylint} est un outil permettant de mesurer la qualité de code selon les conventions de la PEP8 et met une note sur 10
  \end{block}

  A la racine de votre package
\begin{lstlisting}[numbers=none]
python -m pylint mypackage
\end{lstlisting}

  ou simplement
\begin{lstlisting}[numbers=none]
python -m pylint mymodule.py
\end{lstlisting}
    
\end{frame}



\begin{frame}[fragile]{Package \texttt{pylint}}
  \begin{block}{Exemple}
    \medskip
      \only<1>{
        \texttt{python -m pylint sequence\_adn.py}
        \begin{center}
          \includegraphics[width=\textwidth]{img/pylint1.png}
        \end{center}
      }

      \only<2>{
        \texttt{python -m pylint sequence\_adn.py --max-line-length=150}
        \begin{center}
          \includegraphics[width=\textwidth]{img/pylint2.png}
        \end{center}
      }
  \end{block}
    
\end{frame}


\begin{frame}[fragile]{A vous de jouer}

  \begin{enumerate}
    \item Placez-vous à la racine de votre package et exécutez \texttt{pylint}
\begin{lstlisting}[numbers=none]
python -m pylint mypackage --max-line-length=150
\end{lstlisting}
    \item Quelle est votre note sur 10 ?
    \item Améliorez votre code et exécutez à nouveau \texttt{pylint}
  \end{enumerate}
  
\end{frame}

\end{document}