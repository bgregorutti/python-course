\documentclass[10pt]{beamer}



\input{includes}
\title{Langage Python}
\subtitle{M2 Ingénierie Mathématique, Sorbonne Université}
\author{Baptiste Gregorutti\\Ingénieur de Recherche}
\date{Novembre 2024}
\titlegraphic{\hfill\includegraphics[height=1.5cm]{img/logo_sorbonne.png}}


\begin{document}

\maketitle

\begin{frame}{}
  \centering
  \Large
  \textbf{Partie 3}

  \textbf{Librairies scientifiques et visualisation de données}
\end{frame}


\begin{frame}{Most common libraries in Data Science}

  \begin{block}{Data manipulation libraries}
    \begin{itemize}
      \item \texttt{Numpy}: matrices and multi-dimensional arrays
      \item \texttt{Pandas}: data manipulation and analysis
    \end{itemize}
  \end{block}
  \begin{block}{ML library}
    \begin{itemize}
      \item \texttt{Scikit-learn}: most widely used library
    \end{itemize}
  \end{block}
  \begin{block}{Data visualization libraries}
    \begin{itemize}
      \item \texttt{matplotlib}: unavoidable library
      \item \texttt{seaborn}: high-level interface of \texttt{matplotlib}
    \end{itemize}
  \end{block}
\end{frame}


\section{Librairies scientifiques : SciPy, NumPy}

\begin{frame}{Librarie SciPy}
  \begin{center}
    \includegraphics[width=\textwidth]{img/scipy.png}
  \end{center}

\end{frame}

\begin{frame}[fragile]{Librarie NumPy}
  \begin{block}{Librarie dédiées au calcul scientifique}
    \medskip
    \texttt{NumPy} est la librairie de référence pour le calcul scientifique.

    \begin{itemize}
      \item Structure de données représentant des tableaux multidimensionnels \texttt{ndarray}
      \item Un ensemble très complet de routines permettant des opérations optimisées sur des tableaux
      \item Exemples : opérations mathématiques et logiques, tris, algèbre linéaire, opérations statistiques, simulations aléatoires, etc.
    \end{itemize}
  \end{block}

\begin{lstlisting}[language=python, numbers=none, morekeywords=as]
import numpy as np
\end{lstlisting}

\end{frame}




\begin{frame}[fragile]{Structure de données \texttt{ndarray}}

\begin{block}{Création d'un objet \texttt{ndarray} à partir d'un objet itérable}

\begin{lstlisting}[language=python, numbers=none]
arr = np.array(<iterable>)
\end{lstlisting}

\end{block}

Par exemple :
\begin{block}{Création d'un objet \texttt{ndarray} à partir d'un objet itérable}

\begin{lstlisting}[language=python, numbers=none]
>>> arr = np.array([])
>>> arr = np.array([1, 2])
>>> arr = np.array([[1, 2], [3, 4]])
>>> arr = np.array([[[1, 2], [3, 4]], [[5, 6], [7, 8]]])
\end{lstlisting}

\end{block}

{\small
Lien vers la documentation : \href{https://numpy.org/doc/stable/reference/generated/numpy.ndarray.html}{numpy.ndarray.html}}
\end{frame}


\begin{frame}[fragile]{Structure de données \texttt{ndarray}}

  \begin{block}{Indexing et slicing, arrays 1D}

\begin{lstlisting}[language=python, numbers=none]
>>> data = np.array([1, 2, 3])

>>> data[1]
2

>>> data[0:2]
array([1, 2])

>>> data[1:]
array([2, 3])

>>> data[-2:]
array([2, 3])
\end{lstlisting}
  \end{block}
\end{frame}


\begin{frame}[fragile]{Structure de données \texttt{ndarray}}

  \begin{block}{Indexing et slicing, arrays $>$ 2D}

\begin{lstlisting}[language=python, numbers=none]
>>> arr = np.array([[1, 2], [3, 4], [5, 6]])

>>> arr.shape
(3, 2)

>>> arr[0, 1]
2

>>> arr[1:3]
array([[3, 4],
       [5, 6]])

>>> arr[1:3, 0]
array([3, 5])
\end{lstlisting}
  \end{block}
\end{frame}


\begin{frame}{Types de données intégrés à \texttt{NumPy}}

  \begin{itemize}
    \item Entier 64-bits \texttt{np.int64}
    \item Entier 32-bits \texttt{np.int32}
    \item Flottant 32-bits \texttt{np.float32}
    \item Nombre complexe \texttt{np.complex}
    \item Booléen \texttt{np.bool}
    \item Le type de base \texttt{np.object}
  \end{itemize}
\end{frame}


\begin{frame}{Autres fonctionnalités de NumPy}
  \begin{itemize}
    \item Fonctions mathématiques sur les objets \texttt{ndarray}
    \medskip
    \item Simulation de variables aléatoires, module \texttt{np.random}
    \medskip
    \item Importation et stockage de données
  \end{itemize}
\end{frame}


\section{Librairies Pandas pour le traitement de données}



\begin{frame}[fragile]{\includegraphics[width=.15\textwidth]{img/pandas_logo}}
  
  \begin{block}{En bref}
    \medskip
    Pandas est une librairie performante dédiée à la manipulation de données et à l'analyse de données
  \end{block}

  \begin{block}{Liens utiles}
    \begin{itemize}
      \item Documentation officielle : \href{https://pandas.pydata.org/}{pandas.pydata.org}
      \item Pour commencer : \href{https://pandas.pydata.org/getting_started.html}{pandas.pydata.org/getting\_started}
    \end{itemize}
  \end{block}

\begin{lstlisting}[language=python, numbers=none, morekeywords=as]
import pandas as pd
\end{lstlisting}
\end{frame}




\begin{frame}{\includegraphics[width=.15\textwidth]{img/pandas_logo} Structures de données}

  \begin{block}{Deux structures de données}
    \medskip
    \begin{itemize}
      \item \texttt{pd.DataFrame} : tableau bidimensionnel
        \begin{itemize}
          \item Indices sur les lignes et sur les colonnes
          \item Données de types différents
        \end{itemize}
      \item \texttt{pd.Series} : tableau unidimensionnel (une ligne ou une colonne)
    \end{itemize}
  \end{block}

\end{frame}


\begin{frame}{\includegraphics[width=.15\textwidth]{img/pandas_logo} Fonctionnalités de Pandas}

    \begin{itemize}
      \item Operations sur les dataframes
      \medskip 
      \item Importation et sauvegarde des données
      \medskip 
      \item Manipulation des données : exploration, agrégation, concaténation, tri, etc.
      \medskip 
      \item Visualisation
    \end{itemize}
\end{frame}


\section{Machine learning avec Scikit-learn}

  \begin{frame}{Library \texttt{scikit-learn}}

    \centering
    \includegraphics[width=.3\textwidth]{img/logo_sklearn.png}
    \vspace{1cm}
    \begin{block}{Qu'est-ce que \texttt{scikit-learn} ?}
      \medskip
      \begin{itemize}
          \item Bibliothèque pour l'apprentissage automatique
          \item Simplifie l'implémentation d'algorithmes tels que classification, régression, clustering, réduction de dimension, etc.
          \item Compatible avec NumPy, SciPy et matplotlib.
          \item Documentation riche et nombreuses ressources en ligne : \href{https://scikit-learn.org/}{scikit-learn.org}.
      \end{itemize}
    \end{block}
    
  \end{frame}
    
  \begin{frame}{Library \texttt{scikit-learn}}
    \centering
    \includegraphics[width=.3\textwidth]{img/logo_sklearn.png}
    \vspace{1cm}
    \begin{block}{Fonctionnalités}
      \medskip
      \begin{enumerate}
        \item \textbf{Prétraitement} : normalisation des données, encodage, etc.
        \item \textbf{Entraînement} : choix de modèles, entraînements et prédictions
        \item \textbf{Évaluation} : calculer des métriques de performances
      \end{enumerate}
    \end{block}

  \end{frame}


  
  \section{Visualisation de données : Matplotlib}
  
  
  \begin{frame}{\includegraphics[width=.2\textwidth]{img/logo_matplotlib}}
  
    \begin{block}{En bref}
      \medskip
    Matplotlib est LA librairie incontournable pour visualiser des données
    \end{block}
  
    \begin{block}{Liens utiles}
      \begin{itemize}
        \item Site  officiel : \href{https://matplotlib.org/}{matplotlib.org}
        \item {\it Third party packages} : \href{https://matplotlib.org/stable/thirdpartypackages/index.html}{matplotlib.org/stable/thirdpartypackages}
        \item Exemples : \href{https://matplotlib.org/stable/gallery/index.html}{matplotlib.org/stable/gallery}
        \item Tutoriel : \href{https://github.com/matplotlib/AnatomyOfMatplotlib}{github.com/matplotlib/AnatomyOfMatplotlib}
      \end{itemize}
    \end{block}
  
  \end{frame}
  
  
  

  \begin{frame}[fragile]{\includegraphics[width=.2\textwidth]{img/logo_matplotlib}}
  
    \begin{columns}[T]
      \begin{column}{.48\textwidth}
        \begin{block}{Concepts}
          \begin{itemize}
            \item Objet de classe ``Figure''
            \item Objet de classe ``Axes''
            \item Plusieurs graphiques sur une même figure : Subplots
          \end{itemize}
        \end{block}
      \end{column}
  
      \begin{column}{.48\textwidth}
        \begin{center}
          \includegraphics[width=\textwidth]{img/Figure_Axes}
        \end{center}
      \end{column}
    \end{columns}

    \begin{block}{Convention d'import}
\begin{lstlisting}[language=Python, numbers=none]
import matplotlib.pyplot as plt
\end{lstlisting}
    \end{block}

  \end{frame}


  \begin{frame}[fragile]{\includegraphics[width=.2\textwidth]{img/logo_matplotlib.png} Figures, Axes}
  
      \begin{block}{Figure}
          \begin{itemize}
            \item La ``fenêtre'' contenant tous les objets composant le ou les graphiques
            \item Un objet Figure peut contenir : titre, sous-titre, légende, etc.
            \item Une figure est créée comme suit :
          \end{itemize}
      \end{block}
\begin{lstlisting}[language=Python, numbers=none]
fig = plt.figure()
\end{lstlisting}
  
    \begin{block}{Axes}
        \begin{itemize}
          \item Dans une figure, on ajoute un ou plusieurs objets \textcolor{orange}{Axes}
          \item Un tel objet est une \textcolor{orange}{r\'egion} de l'image qui recevra les \textcolor{orange}{donn\'es à visualiser}
          \item Une figure peut contenir plusieurs axes
          \item Un axe contient deux ou trois objets \textcolor{orange}{Axis}
        \end{itemize}
    \end{block}
  \end{frame}


  
  
  \begin{frame}[fragile]{\includegraphics[width=.2\textwidth]{img/logo_matplotlib} Figures and Axes}
\begin{lstlisting}[language=Python, morekeywords={as}, numbers=none]
import matplotlib.pyplot as plt

fig = plt.figure() # Create the Figure object

# Add an Axe to this figure
ax = fig.add_axes([0.5,0.5,0.8,0.8]) # dimensions [left, bottom, width, height] of the new axes

# Plot the distribution of the wind variable in the axe
g1 = ax.hist(x=seattle_weather['wind'])

# Add a title and labels for the x-axis and the y-axis
ax.set_title("Histogram with default parameters")
ax.set_xlabel("Wind")
ax.set_ylabel("Count")

plt.show() # Show the plot
\end{lstlisting}
  \end{frame}
  
  \begin{frame}[fragile]{\includegraphics[width=.2\textwidth]{img/logo_matplotlib} Figures and Axes}
\begin{lstlisting}[language=Python, morekeywords={as}, numbers=none]
# Short version (using the fact that figure and axes are automatically created)

import matplotlib.pyplot as plt

# Plot the distribution of the wind using a histogram
plt.hist(x=seattle_weather["wind"])

# Add a title
plt.title("Histogram with default parameters")

# Add labels for the x-axis and the y-axis
plt.xlabel("Wind")
plt.ylabel("Count")

plt.show() # Show the plot
\end{lstlisting}
  \end{frame}
  
  \begin{frame}{\includegraphics[width=.2\textwidth]{img/logo_matplotlib} Subplots}
    \begin{block}{What is a subplot?}
      \begin{itemize}
        \item The matplotlib.pyplot.subplots method provides a way to have \textcolor{orange}{multiple plots on a single figure}
        \item Given the number of rows and columns, it returns a tuple (fig, ax), giving a single figure `fig` with an array of axe `ax`
      \end{itemize}
    \end{block}
    More details:
    \begin{itemize}
      \item \href{https://www.educative.io/edpresso/what-is-a-subplots-in-matplotlib}{educative.io/edpresso/what-is-a-subplots-in-matplotlib}
      \item \href{https://matplotlib.org/stable/gallery/subplots_axes_and_figures/subplots_demo.html}{matplotlib.org/stable/gallery/subplots\_axes\_and\_figures}
    \end{itemize}
  \end{frame}
  
  
  \begin{frame}[fragile]{\includegraphics[width=.2\textwidth]{img/logo_matplotlib} Subplots}
  
    \begin{block}{Usage 1}
\begin{lstlisting}[language=Python, morekeywords={as}, numbers=none]
import matplotlib.pyplot as plt

# Create a subplot with 1 row and 2 columns
fig, (ax1, ax2) = plt.subplots(1, 2)

# Hist on the 1st column
ax1.hist(x)

# Scatter on the 2nd column
ax2.scatter(x, y)

# Show the plot
plt.show()
\end{lstlisting}
    \end{block}
  
  \end{frame}
  
  \begin{frame}[fragile]{\includegraphics[width=.2\textwidth]{img/logo_matplotlib} Subplots}
  
    \begin{block}{Usage 2}
\begin{lstlisting}[language=Python, morekeywords={as}, numbers=none]
import matplotlib.pyplot as plt

# Create the figure
fig = plt.figure()

# Create an histogram on the 1st subplot
plt.subplot(121) # or plt.subplot(1, 2, 1)
plt.hist(x)

# Create scatter plot on the 2nd subplot
plt.subplot(122) # or plt.subplot(1, 2, 2)
plt.scatter(x, y)

# Show the plot
plt.show()
\end{lstlisting}
    \end{block}
  
  \end{frame}



  \begin{frame}{Seaborn: high-level interface of matplotlib}

    \begin{block}{Overview of seaborn}
      \medskip
      Seaborn is a \textcolor{blue}{complementary} library for Matplotlib, which provides a high-level \textcolor{blue}{interface} to draw \textcolor{blue}{statistical} graphics.
  
      It uses Pandas data frames
    \end{block}
  
    \begin{block}{Useful links}
      \medskip
      \begin{itemize}
        \item Gallery: \href{https://seaborn.pydata.org/examples/index.html}{seaborn.pydata.org/examples}
        \item Tutorial: \href{https://seaborn.pydata.org/tutorial.html}{seaborn.pydata.org/tutorial}
      \end{itemize}
      
    \end{block}
  \end{frame}
  
  
  
  \begin{frame}{Seaborn: high-level interface of matplotlib}
  
    \begin{columns}[T]
      \begin{column}{.25\textwidth}
       \medskip
       \centering
       \textsc{Figure-level}
       
       vs
       
       \textsc{Axes-level}
      \end{column}
  
      \begin{column}{.25\textwidth}
        \medskip
        \centering
        \textsc{Relational}
  
        \textsc{Distributional}
  
        \textsc{Categorical}
      \end{column}
  
    \end{columns}
  
    \centering
    \includegraphics[width=.7\textwidth]{/Users/bgregorutti/Documents/pro/SCAI/Formation/AbuDhabi/materials/data-visualization-slides/dataviz_python/img/seaborn-overview.png}
  \end{frame}
  
  
  
  
  \begin{frame}[fragile,t]{Figure-level vs Axes-level functions}
  
    \begin{block}{Axes-level}
      \medskip
      Axes-level functions plots data onto a single matplotlib Axes object ax
    \end{block}
  
\begin{lstlisting}[language=python, numbers=none]
# creates a chart in an existing axes object
sns.histplot(data=df, x="temperature", ax=ax)
\end{lstlisting}
  
    \onslide<2>{
      \begin{block}{Figure-level}
        \medskip
        Figure-level functions interface with matplotlib through a seaborn object that manages the figure.
        
        For each task, the figure-level function is an interface to various axes-level functions
        
        % For example, the figure-level function `displot` is an interface to the axes-level functions `histplot`, `kdeplot`, etc. see the figure below.
      \end{block}
    }
    \begin{onlyenv}<2>
\begin{lstlisting}[language=python, numbers=none]
sns.displot(data=df, x="temperature", kind="hist")
\end{lstlisting}    
    \end{onlyenv}
  \end{frame}
  
  
  
  
  \begin{frame}{Figure-level vs Axes-level functions}
  
    \begin{columns}[T]
      \begin{column}{.25\textwidth}
       \medskip
       \centering
       \textsc{Figure-level}
       
       vs
       
       \textsc{Axes-level}
      \end{column}
  
      \begin{column}{.25\textwidth}
  
      \end{column}
  
    \end{columns}
  
    \centering
    \includegraphics[width=.7\textwidth]{/Users/bgregorutti/Documents/pro/SCAI/Formation/AbuDhabi/materials/data-visualization-slides/dataviz_python/img/seaborn-overview.png}
  \end{frame}
  
  
  \begin{frame}[fragile]{Same usage / Different tasks}
  
    \begin{block}{The functions have (almost) the same signature}
      \medskip
      \begin{itemize}
        \item data: a pandas DataFrame
        \item x, y: columns names in the data frame
        \item hue: name of the column used to add colors
        \item ax: maplotlib Axes object in which to draw the plot
      \end{itemize}
      
    \end{block}
  
\begin{lstlisting}[language=python, numbers=none]
# Scatter plot
sns.scatterplot(data=df, x="time", y="temperature", ax=ax)

# Line plot
sns.lineplot(data=df, x="time", y="temperature", ax=ax)

# Line plot, Figure-level function
sns.relplot(data=df, x="time", y="temperature", kind="line")
\end{lstlisting}
  
  \end{frame}
  
  \begin{frame}[fragile]{Statistical graphics}
  
    Seaborn integrates statistics into the graphics making easier the analysis of the data.
  
    \begin{itemize}
      \item Automatic computation of statistics: mean, confidence interval, error bars, etc.
      \item Automatic linear regression fitting
    \end{itemize}
    
    \begin{center}
      \includegraphics[width=.4\textwidth]{/Users/bgregorutti/Documents/pro/SCAI/Formation/AbuDhabi/materials/data-visualization-slides/dataviz_python/img/cat-error-bar.png}
      \includegraphics[width=.4\textwidth]{/Users/bgregorutti/Documents/pro/SCAI/Formation/AbuDhabi/materials/data-visualization-slides/dataviz_python/img/reg-fit.png}
    \end{center}
  
  \end{frame}
  
  \begin{frame}[fragile]{Figure aesthetics}
    The aesthetics can be modified, helping the communication of insights
  
    The function \texttt{set\_theme} modifes the background, the color palette, the font, etc.

\begin{lstlisting}[language=python, numbers=none]
# Default theme
sns.set_theme()

# Dark background with a grid
sns.set_theme(style="darkgrid")

# White background with a grid
sns.set_theme(style="whitegrid", palette="pastel")
\end{lstlisting}
    
    \begin{center}
      \includegraphics[width=.3\textwidth]{/Users/bgregorutti/Documents/pro/SCAI/Formation/AbuDhabi/materials/data-visualization-slides/dataviz_python/img/set-theme-1.png}
      \includegraphics[width=.3\textwidth]{/Users/bgregorutti/Documents/pro/SCAI/Formation/AbuDhabi/materials/data-visualization-slides/dataviz_python/img/set-theme-2.png}
      \includegraphics[width=.3\textwidth]{/Users/bgregorutti/Documents/pro/SCAI/Formation/AbuDhabi/materials/data-visualization-slides/dataviz_python/img/set-theme-3.png}
    \end{center}
  
  \end{frame}


  
  
  
  \begin{frame}[fragile]{Some examples}
  
    \begin{block}{Plotting distributions}
\begin{lstlisting}[language=python, numbers=none]
# Axes-level histogram
sns.histplot(x="value", hue="variable", data=df)

# Figure-level histogram
sns.displot(x="value", hue="variable", data=df)
\end{lstlisting}
    \end{block}
  
    \begin{center}
      \includegraphics[width=.4\textwidth]{img/hist_axe.png}
      \includegraphics[width=.4\textwidth]{img/hist_figure.png}
    \end{center}
  
  \end{frame}
  
\begin{frame}[fragile]{Some examples}
  
  \begin{block}{Plotting distributions}
\begin{lstlisting}[language=python, numbers=none]
# Axes-level histogram
sns.kdeplot(x="value", hue="variable", data=df)

# Figure-level histogram
sns.displot(x="value", hue="variable", data=df, kind="kde")
\end{lstlisting}
  \end{block}
  
  \begin{center}
    \includegraphics[width=.4\textwidth]{img/kde_axe.png}
    \includegraphics[width=.4\textwidth]{img/kde_figure.png}
  \end{center}

\end{frame}

  
  
  \begin{frame}[fragile]{Some examples}
  
    \begin{block}{}
\begin{lstlisting}[language=python, numbers=none]
# Pairplot / Matrix of scatterplots
sns.pairplot(x)

# Joint plot
sns.jointplot(x)
\end{lstlisting}
    \end{block}
  
    \begin{center}
      \includegraphics[width=.4\textwidth]{img/scatterplot_matrix}
      \includegraphics[width=.4\textwidth]{img/joint}
    \end{center}
  
\end{frame}


    
\begin{frame}{Seaborn: high-level interface of matplotlib}
  \begin{center}
    \includegraphics[width=\textwidth]{/Users/bgregorutti/Documents/pro/SCAI/Formation/AbuDhabi/materials/data-visualization-slides/dataviz_python/img/seaborn-cheat.png}
  \end{center}
  \tiny{Source: interactivechaos.com}
\end{frame}




\end{document}