\documentclass[10pt]{beamer}

\input{includes}
\title{Langage Python}
\subtitle{M2 Ingénierie Mathématique, Sorbonne Université}
\author{Baptiste Gregorutti\\Ingénieur de Recherche}
\date{Novembre 2024}
\titlegraphic{\hfill\includegraphics[height=1.5cm]{img/logo_sorbonne.png}}


\begin{document}

\maketitle

\begin{frame}{}
  \centering
  \Large
  \textbf{Partie 3}

  \textbf{Librairies scientifiques et visualisation de données}
\end{frame}


\begin{frame}{Most common libraries in Data Science}

  \begin{block}{Data manipulation libraries}
    \begin{itemize}
      \item \texttt{Numpy}: matrices and multi-dimensional arrays
      \item \texttt{Pandas}: data manipulation and analysis
    \end{itemize}
  \end{block}
  \begin{block}{ML library}
    \begin{itemize}
      \item \texttt{Scikit-learn}: most widely used library
    \end{itemize}
  \end{block}
  \begin{block}{Data visualization libraries}
    \begin{itemize}
      \item \texttt{matplotlib}: unavoidable library
      \item \texttt{seaborn}: high-level interface of \texttt{matplotlib}
    \end{itemize}
  \end{block}
\end{frame}


\section{Librairies scientifiques : SciPy, NumPy}

\begin{frame}[fragile]{Qu'est-ce que NumPy ?}

\textbf{NumPy} (Numerical Python) est la bibliothèque fondamentale pour le calcul scientifique en Python.

\medskip

\textbf{Caractéristiques principales :}
\begin{itemize}
    \item Objet tableau N-dimensionnel puissant (\texttt{ndarray})
    \item Opérations mathématiques efficaces sur les tableaux
    \item Algèbre linéaire, transformées de Fourier, nombres aléatoires
    \item Base de pandas, scikit-learn et la plupart des bibliothèques scientifiques
\end{itemize}

\medskip

\begin{lstlisting}[language=python, numbers=none]
import numpy as np

# Creer un tableau NumPy
arr = np.array(<iterable>)
\end{lstlisting}

\end{frame}







\begin{frame}[fragile]{Structure de données \texttt{ndarray}}

\textbf{Exemple :} Création d'un objet \texttt{ndarray} à partir d'un objet itérable, ici une liste
\bigskip
\begin{lstlisting}[language=python, numbers=none]
>>> arr = np.array([])
>>> arr = np.array([1, 2])
>>> arr = np.array([[1, 2], [3, 4]])
>>> arr = np.array([[[1, 2], [3, 4]], [[5, 6], [7, 8]]])
\end{lstlisting}

Tableau de dimension 1 : \textbf{vecteur}

Tableau de dimension 2 : \textbf{matrice}

Tableau de dimension supérieure à 2 : \textbf{tenseur}.


\end{frame}


\begin{frame}[fragile]{Structure de données \texttt{ndarray}}

Différentes façons de créer des tableaux :

\begin{lstlisting}[language=python, numbers=none]
# Tableaux speciaux
zeros = np.zeros((3, 4))        
ones = np.ones((2, 3))          
identity = np.eye(4)            

# Sequences
arange = np.arange(0, 10, 2)    
linspace = np.linspace(0, 1, 5) 

# Aleatoires
random = np.random.rand(3, 3)   
randn = np.random.randn(3, 3)   
\end{lstlisting}

\end{frame}








\begin{frame}[fragile]{Structure de données \texttt{ndarray}}

  \begin{block}{Indexing et slicing}
    \medskip
    L'accès aux éléments d'un tableau NumPy se fait \textbf{par indices}:
\begin{lstlisting}[language=python, numbers=none]
# 1D array, vecteur
arr[start:end:step]

# 2D array, matrice
arr[start:end:step, start:end:step]

# 3D array, tenseur
arr[start:end:step, start:end:step, start:end:step]
\end{lstlisting}
  \end{block}
\end{frame}



\begin{frame}[fragile]{Structure de données \texttt{ndarray}}
  \textbf{Exemples :}

\begin{lstlisting}[language=python, numbers=none]
# Tableau 2D (matrice)
matrice = np.array([[1, 2, 3],
                    [4, 5, 6]])
print(matrice.shape)  # (2, 3)

# Tableau 3D (tenseur)
tenseur = np.zeros((3, 4, 5))  # 3 couches, 4 lignes, 5 cols

# Indexation
print(matrice[0, 1])      
print(matrice[:, 0])      
print(matrice[1, :])      
\end{lstlisting}


\end{frame}







\begin{frame}[fragile]
\frametitle{Listes vs tableaux NumPy}

\begin{columns}
\column{0.5\textwidth}
\textbf{Listes}
\begin{itemize}
    \item Conteneur générique
    \item Peut contenir des types mixtes
    \item Flexible (append, insert, etc.)
    \item Plus lent pour les calculs
\end{itemize}

\begin{lstlisting}[language=python, numbers=none]
lst = [1, 2, 3, 4]
mixed = [1, "bonjour", 3.14]
\end{lstlisting}

\column{0.5\textwidth}



\end{columns}

\end{frame}





\begin{frame}[fragile]
\frametitle{Listes vs tableaux NumPy}

\begin{columns}
\column{0.5\textwidth}
\textbf{Listes}
\begin{itemize}
    \item Conteneur générique
    \item Peut contenir des types mixtes
    \item Flexible (append, insert, etc.)
    \item Plus lent pour les calculs
\end{itemize}

\begin{lstlisting}[language=python, numbers=none]
lst = [1, 2, 3, 4]
mixed = [1, "bonjour", 3.14]
\end{lstlisting}

\column{0.5\textwidth}
\textbf{Tableaux NumPy}
\begin{itemize}
    \item Optimisé pour les données numériques
    \item Type fixe (homogène)
    \item Taille fixe (redimensionnement possible)
    \item Beaucoup plus rapide
\end{itemize}

\begin{lstlisting}[language=python, numbers=none]
# Tableau NumPy
arr = np.array([1, 2, 3, 4])
\end{lstlisting}

\end{columns}

\end{frame}

\begin{frame}[fragile]
\frametitle{Comparaison de performance}

\textbf{Pourquoi NumPy est-il plus rapide ?}

\begin{itemize}
    \item Écrit en C (code compilé)
    \item Stockage contigu en mémoire
    \item \textcolor{red}{Opérations vectorisées} (pas de boucles Python)
\end{itemize}

\vspace{0.5cm}

\begin{lstlisting}[language=python, numbers=none]
# Liste Python : element par element (lent)
lst = [1, 2, 3, 4, 5]
result = [x * 2 for x in lst]

# NumPy : operation vectorisee (rapide !)
arr = np.array([1, 2, 3, 4, 5])
result = arr * 2
\end{lstlisting}

\vspace{0.3cm}
\textbf{Différence de vitesse :} NumPy peut être 10 à 100x plus rapide pour de grandes données !

\end{frame}


\begin{frame}[fragile]{Opérations : listes vs NumPy}

\begin{lstlisting}[language=python, numbers=none]
# Listes Python - necessite des boucles
liste1 = [1, 2, 3]
liste2 = [4, 5, 6]
resultat = [a + b for a, b in zip(liste1, liste2)]

# NumPy - operations vectorisees
arr1 = np.array([1, 2, 3])
arr2 = np.array([4, 5, 6])
resultat = arr1 + arr2  # Addition element par element

# Operations mathematiques
arr = np.array([1, 2, 3, 4])
print(np.mean(arr))      
print(np.std(arr))       
print(np.sum(arr))       
\end{lstlisting}

\end{frame}

\begin{frame}[fragile]{Algèbre linéaire}

NumPy fournit des opérations matricielles efficaces :

\begin{lstlisting}[language=python, numbers=none]
# Creer des matrices
A = np.array([[1, 2], [3, 4]])
B = np.array([[5, 6], [7, 8]])

# Multiplication matricielle
C = A @ B  # ou np.dot(A, B)

# Operations matricielles
det = np.linalg.det(A)        
inv = np.linalg.inv(A)        
valeurs_propres = np.linalg.eig(A)  

# Transposee
A_T = A.T
\end{lstlisting}

\end{frame}



\begin{frame}[fragile]{Broadcasting : le super pouvoir de NumPy}

Le broadcasting permet des opérations entre tableaux de formes différentes :

\begin{lstlisting}[language=python, numbers=none]
# Ajouter un scalaire a un tableau
arr = np.array([1, 2, 3, 4])
resultat = arr + 10

# Ajouter un vecteur 1D a chaque ligne d'une matrice 2D
matrice = np.array([[1, 2, 3],
                    [4, 5, 6]])
vecteur = np.array([10, 20, 30])
resultat = matrice + vecteur
\end{lstlisting}

\small
Le broadcasting élimine le besoin de boucles explicites et est très efficace !

\end{frame}



\begin{frame}[fragile]{Quand utiliser chaque structure}

\begin{table}
\small
\begin{tabular}{|l|l|l|}
\hline
\textbf{Structure} & \textbf{Utiliser quand...} & \textbf{Exemple} \\
\hline
\textbf{Liste} & Types mixtes nécessaires & \texttt{['nom', 42, True]} \\
 & Insertions fréquentes & Collections dynamiques \\
 & Petits ensembles & Données de config \\
\hline
\textbf{NumPy} & Calculs numériques & Opérations matricielles \\
 & Grands ensembles & Données scientifiques \\
 & Opérations vectorisées & Traitement d'images \\
 & Opérations mathématiques & Statistiques, ML \\
\hline
\end{tabular}
\end{table}

\vspace{0.5cm}
\textbf{Règle générale :} NumPy pour les nombres, listes pour tout le reste !

\end{frame}


\begin{frame}[fragile]{Comparaison avec d'autres structures}

\begin{table}
\begin{tabular}{|l|c|c|c|c|}
\hline
\textbf{Critère} & \textbf{Liste} & \textbf{Tuple} & \textbf{NumPy} & \textbf{Dict} \\
\hline
Mutable & Oui & Non & Oui & Oui \\
\hline
Types mixtes & Oui & Oui & \textcolor{red}{Non} & Oui \\
\hline
Opérations math & Lent & Lent & \textcolor{blue}{Rapide} & Non \\
\hline
Accès élément & O(1) & O(1) & O(1) & O(1) \\
\hline
Mémoire & Élevée & Moyenne & \textcolor{blue}{Faible} & Élevée \\
\hline
Multi-dim & Non & Non & \textcolor{blue}{Oui} & Non \\
\hline
\end{tabular}
\end{table}

\end{frame}



\begin{frame}
\frametitle{Résumé}

\textbf{Points clés :}

\begin{itemize}
    \item Les tableaux NumPy sont optimisés pour les calculs numériques
    \item Beaucoup plus rapides que les listes Python pour les opérations mathématiques
    \item Type fixe, mémoire contiguë, opérations vectorisées
    \item Essentiel pour le calcul scientifique, l'analyse de données et le ML
    \item Utilisez NumPy pour les nombres, les listes pour les collections génériques
    \item Le broadcasting permet des opérations puissantes \textcolor{red}{sans boucles}
\end{itemize}

\end{frame}








\section{Librairies Pandas pour le traitement de données}




\begin{frame}[fragile]{Qu'est-ce que Pandas ?}

\textbf{Pandas} est une bibliothèque très largement utilisée dans le domaine de l'analyse des données.

\vspace{0.5cm}

\textbf{Caractéristiques principales :}
\begin{itemize}
    \item Manipulation et analyse de \textbf{données tabulaires}
    \item Structures de données adaptées (Series et DataFrame)
    \item Intégration avec NumPy, Matplotlib, Seaborn
    \item Lecture/écriture de multiples formats de fichiers
    \item Opérations statistiques intégrées
\end{itemize}

\vspace{0.5cm}

\textbf{Installation et importation :}
\begin{lstlisting}[language=python, numbers=none]
import pandas as pd
\end{lstlisting}

\end{frame}

\begin{frame}{Structures de données Pandas}

Pandas repose sur deux structures principales :

\vspace{0.5cm}

\begin{columns}
\column{0.5\textwidth}
\textbf{Series}
\begin{itemize}
    \item Tableau 1D
    \item Similaire à une colonne/ligne
    \item Types de données \textbf{mixtes}
    \item Index personnalisable
\end{itemize}

\column{0.5\textwidth}
\textbf{DataFrame}
\begin{itemize}
    \item Tableau 2D
    \item Lignes = observations
    \item Colonnes = variables
    \item Chaque colonne/ligne est une Series
\end{itemize}
\end{columns}

\vspace{0.5cm}
\onslide<2>
\textbf{Différences avec NumPy :}
\begin{itemize}
    \item Pandas : maximum 2 dimensions
    \item Types mixtes par colonne (vs homogène pour NumPy)
    \item Accès aux éléments par index et labels
\end{itemize}

\end{frame}

\begin{frame}[fragile]{Series : Tableau 1D}

Une série est un tableau unidimensionnel avec index.

\begin{lstlisting}[language=python, numbers=none]
data = [1, 2, 3, 4, 5]
serie = pd.Series(data)
\end{lstlisting}

\textbf{Attributs principaux :}
\begin{itemize}
    \item \texttt{data} : les données à stocker
    \item \texttt{index} : indices personnalisés (défaut : 0, 1, 2, ...)
    \item \texttt{dtype} : type des données (inféré si non spécifié)
    \item \texttt{name} : nom de la série
\end{itemize}

\begin{lstlisting}[language=python, numbers=none]
serie = pd.Series(
    data=[1, 2, 3], 
    index=["a", "b", "c"],
    name="exemple"
)
\end{lstlisting}

\end{frame}

\begin{frame}[fragile]{Accès aux éléments d'une series}

\textbf{Accès par index :}
\begin{lstlisting}[language=python, numbers=none]
serie = pd.Series([10, 20, 30], index=["a", "b", "c"])

# Acces direct
print(serie["a"])

# Slicing
print(serie["a":"c"])

# Avec .loc
print(serie.loc["a"])
print(serie.loc["a":"b"])
\end{lstlisting}

\textbf{Opérations mathématiques :}
\begin{lstlisting}[language=python, numbers=none]
serie = pd.Series([1, 2, 3, 4, 5])

print(serie + serie)  # Addition element par element
print(serie * 2)      # Multiplication par scalaire
print(serie ** 2)     # Puissance
\end{lstlisting}

\end{frame}

\begin{frame}[fragile]{DataFrame : tableau 2D}

Une DataFrame est une structure 2D similaire à une table.

\textbf{Création à partir d'une liste :}
\begin{lstlisting}[language=python, numbers=none]
data = [('Alice', 25), ('Bob', 30), ('Charlie', 35)]
df = pd.DataFrame(
    data=data, 
    columns=("name", "age"),
    index=("item_1", "item_2", "item_3")
)
\end{lstlisting}

\textbf{Création à partir d'un dictionnaire :}
\begin{lstlisting}[language=python, numbers=none]
data = {
    'name': ['Alice', 'Bob', 'Charlie'], 
    'age': [25, 30, 35]
}
df = pd.DataFrame(
    data=data, 
    index=("item_1", "item_2", "item_3")
)
\end{lstlisting}

\end{frame}

\begin{frame}[fragile]{Exploration d'une DataFrame}

\textbf{Afficher les caractéristiques :}
\begin{lstlisting}[language=python, numbers=none]
print(df.index)     # Labels des lignes
print(df.columns)   # Noms des colonnes
print(df.shape)     # Dimensions (lignes, colonnes)
print(df.info())    # Informations generales
print(df.dtypes)    # Types de chaque colonne
\end{lstlisting}

\textbf{Aperçu des données :}
\begin{lstlisting}[language=python, numbers=none]
print(df.head())      # Premieres lignes
print(df.tail())      # Dernieres lignes
print(df.describe())  # Statistiques descriptives
\end{lstlisting}

\end{frame}

\begin{frame}[fragile]{Accès aux éléments d'une DataFrame}

\textbf{Cas 1 : Sélection d'une colonne}
\begin{lstlisting}[language=python, numbers=none]
df["name"]      # Renvoie une Series
df.name         # Alternative (si pas d'espaces dans le nom)
\end{lstlisting}

\textbf{Cas 2 : Sélection d'une ligne}
\begin{lstlisting}[language=python, numbers=none]
df.loc["item_1", :]  # Renvoie une Series
\end{lstlisting}

\textbf{Cas 3 : Sélection d'un élément}
\begin{lstlisting}[language=python, numbers=none]
df.loc["item_1", "name"]  # Valeur unique
\end{lstlisting}

\textbf{Slicing :}
\begin{lstlisting}[language=python, numbers=none]
df.loc["item_1":"item_3", "name"]
df.loc[:, ["name", "age"]]
\end{lstlisting}

\end{frame}

\begin{frame}[fragile]{Opérations mathématiques}

Les opérations s'effectuent élément par élément :

\begin{lstlisting}[language=python, numbers=none]
data = {
    'name': ['Alice', 'Bob', 'Charlie'], 
    'age': [30, 25, 35]
}
df = pd.DataFrame(data=data)

print(df + df)  # Fonctionne si operations definies
print(df * 3)
\end{lstlisting}

\vspace{0.3cm}

\textbf{Attention :} Les opérations ne s'appliquent que si elles sont définies pour les types concernés. Python n'effectuera pas le calcul sur les colonnes incompatibles.

\end{frame}

\begin{frame}[fragile]{Fonctions statistiques}

Pandas offre de nombreuses méthodes statistiques :

\begin{lstlisting}[language=python, numbers=none]
df.min()      # Minimum
df.max()      # Maximum
df.mean()     # Moyenne
df.median()   # Mediane
df.std()      # Ecart-type
df.var()      # Variance
df.cov()      # Covariance
df.corr()     # Correlation
\end{lstlisting}

\textbf{Arguments communs :}
\begin{itemize}
    \item \texttt{axis} : calcul sur lignes (0) ou colonnes (1)
    \item \texttt{skipna} : exclure les valeurs manquantes ?
    \item \texttt{numeric\_only} : colonnes numériques uniquement ?
\end{itemize}

\end{frame}


\begin{frame}[fragile]{Chargement de données}

Pandas peut lire de nombreux formats de fichiers :

\begin{lstlisting}[language=python, numbers=none]
# CSV
df = pd.read_csv("donnees.csv", sep=",")

# Excel
df = pd.read_excel("donnees.xlsx")

# JSON
df = pd.read_json("donnees.json")

# SQL
df = pd.read_sql(query, connection)

# Fichier zippe (sans decompression)
df = pd.read_csv("donnees.zip")
\end{lstlisting}

\textbf{Avantages :}
\begin{itemize}
    \item Types de données inférés automatiquement
    \item Noms de colonnes détectés
    \item Support des URL distantes
\end{itemize}

\end{frame}

\begin{frame}[fragile]{Sauvegarde de données}

Sauvegarder une DataFrame dans différents formats :

\begin{lstlisting}[language=python, numbers=none]
# CSV
df.to_csv("donnees_modifiees.csv", index=False)

# Excel
df.to_excel("donnees.xlsx", index=False)

# JSON
df.to_json("donnees.json")

# SQL
df.to_sql(table_name, connection)
\end{lstlisting}

\vspace{0.5cm}

L'argument \texttt{index=False} évite de sauvegarder les indices si ce sont de simples incréments (0, 1, 2, ...).

\end{frame}

\begin{frame}[fragile]{Tri des données}

Trier une DataFrame selon une ou plusieurs colonnes :

\begin{lstlisting}[language=python, numbers=none]
df = pd.DataFrame({
    "name": ["Alice", "Bob", "Charlie"], 
    "age": [30, 25, 35]
})

# Trier par age
df_sorted = df.sort_values(by="age")
print(df_sorted)

# Trier par plusieurs colonnes
df_sorted = df.sort_values(by=["age", "name"])

# Ordre decroissant
df_sorted = df.sort_values(by="age", ascending=False)
\end{lstlisting}

\end{frame}

\begin{frame}[fragile]{GroupBy et agrégation}

La méthode \texttt{groupby} permet de grouper et agréger les données :

\begin{lstlisting}[language=python, numbers=none]
df = pd.DataFrame({
    "Animal": ["Falcon", "Falcon", "Parrot", "Parrot"], 
    "Max Speed": [380., 370., 24., 26.]
})

# Grouper par Animal
gb = df.groupby(["Animal"])

# Agregation avec statistiques
print(gb.mean())    # Moyenne par groupe
print(gb.min())     # Minimum par groupe
print(gb.max())     # Maximum par groupe
print(gb.count())   # Nombre d'elements par groupe
print(gb.sum())     # Somme par groupe
\end{lstlisting}

\end{frame}


\begin{frame}[fragile]{Concaténation de DataFrames}

Combiner plusieurs DataFrames avec \texttt{pd.concat()} :

\begin{lstlisting}[language=python, numbers=none]
df1 = pd.DataFrame({
    "col1": ["a", "b", "c"], 
    "col2": [1, 2, 3]
})
df2 = pd.DataFrame({
    "col1": ["d", "e", "f"], 
    "col2": [4, 5, 6]
})

# Concatenation en lignes (par defaut)
result = pd.concat([df1, df2])
result = pd.concat([df1, df2], axis=0)

# Ignorer les anciens indices
result = pd.concat([df1, df2], ignore_index=True)

# Concatenation en colonnes
result = pd.concat([df1, df2], axis=1)
\end{lstlisting}

\end{frame}

\begin{frame}{Formats \texttt{wide} vs \texttt{long}}

\textbf{Format Wide (large) :}
\begin{itemize}
    \item Lignes = individus uniques
    \item Colonnes = caractéristiques distinctes
    \item Aussi appelé : table pivot, tableau croisé
\end{itemize}

\vspace{0.3cm}

\begin{table}
\footnotesize
\begin{tabular}{lccc}
\toprule
Person & Age & Weight & Height \\
\midrule
Bob & 32 & 168 & 180 \\
Alice & 24 & 150 & 175 \\
Steve & 64 & 144 & 165 \\
\bottomrule
\end{tabular}
\end{table}

\vspace{0.3cm}

\textbf{Format Long (stacked) :}
\begin{itemize}
    \item 3 colonnes : identifiant, variable, valeur
    \item Plus compact pour données répétitives
\end{itemize}

\end{frame}

\begin{frame}{Formats \texttt{wide} vs \texttt{long} (suite)}

\textbf{Format Long :}

\begin{table}
\footnotesize
\begin{tabular}{lcc}
\toprule
Person & Variable & Value \\
\midrule
Bob & Age & 32 \\
Bob & Weight & 168 \\
Bob & Height & 180 \\
Alice & Age & 24 \\
Alice & Weight & 150 \\
Alice & Height & 175 \\
Steve & Age & 64 \\
Steve & Weight & 144 \\
Steve & Height & 165 \\
\bottomrule
\end{tabular}
\end{table}

\end{frame}

\begin{frame}[fragile]{Conversion \texttt{wide} / \texttt{long}}

\textbf{Long → Wide avec \texttt{pivot()} :}
\begin{lstlisting}[language=python, numbers=none]
df_long = pd.DataFrame({
    "Person": ["Bob", "Bob", "Alice", "Alice"],
    "Variable": ["Age", "Weight", "Age", "Weight"],
    "Value": [32, 168, 24, 150]
})

df_wide = df_long.pivot(
    index="Person", 
    columns="Variable", 
    values="Value"
)
\end{lstlisting}

\textbf{Wide → Long avec \texttt{melt()} :}
\begin{lstlisting}[language=python, numbers=none]
# Conserver l'index
df_long = pd.melt(
    df_wide.reset_index(), 
    id_vars=["Person"]
)
\end{lstlisting}

\end{frame}

\begin{frame}[fragile]{Visualisation avec Pandas}

Pandas s'intègre avec Matplotlib et Seaborn :

\begin{lstlisting}[language=python, numbers=none]
import pandas as pd
import matplotlib.pyplot as plt

df = pd.DataFrame({
    "name": ["Alice", "Bob", "Charlie"], 
    "age": [30, 25, 35]
})

# Histogramme
df['age'].plot(kind='hist', bins=10)
plt.show()

# Autres types de graphiques
df.plot(kind='bar')      # Diagramme en barres
df.plot(kind='line')     # Courbe
df.plot(kind='scatter', x='col1', y='col2')  # Nuage
\end{lstlisting}

\end{frame}

\begin{frame}{Résumé}

\textbf{Points clés :}

\begin{itemize}
    \item \textbf{Series} : tableaux 1D avec index
    \item \textbf{DataFrame} : tableaux 2D (lignes x colonnes)
    \item Accès via index/labels avec \texttt{.loc}
    \item Opérations mathématiques et statistiques intégrées
    \item Lecture/écriture de multiples formats (CSV, Excel, JSON, SQL)
    \item GroupBy pour agrégations
    \item Concaténation et réorganisation (pivot/melt)
    \item Intégration avec NumPy et Matplotlib
\end{itemize}

\vspace{0.5cm}

\textbf{Ressource :} Pandas Cheat Sheet disponible sur pandas.pydata.org

\end{frame}



  
  \section{Visualisation de données : Matplotlib}
  
  
  \begin{frame}{\includegraphics[width=.2\textwidth]{img/logo_matplotlib}}
  
    \begin{block}{En bref}
      \medskip
    Matplotlib est LA librairie incontournable pour visualiser des données
    \end{block}
  
    \begin{block}{Liens utiles}
      \begin{itemize}
        \item Site  officiel : \href{https://matplotlib.org/}{matplotlib.org}
        \item {\it Third party packages} : \href{https://matplotlib.org/stable/thirdpartypackages/index.html}{matplotlib.org/stable/thirdpartypackages}
        \item Exemples : \href{https://matplotlib.org/stable/gallery/index.html}{matplotlib.org/stable/gallery}
        \item Tutoriel : \href{https://github.com/matplotlib/AnatomyOfMatplotlib}{github.com/matplotlib/AnatomyOfMatplotlib}
      \end{itemize}
    \end{block}
  
  \end{frame}
  
  
  

  \begin{frame}[fragile]{\includegraphics[width=.2\textwidth]{img/logo_matplotlib}}
  
    \begin{columns}[T]
      \begin{column}{.48\textwidth}
        \begin{block}{Concepts}
          \begin{itemize}
            \item Objet de classe ``Figure''
            \item Objet de classe ``Axes''
            \item Plusieurs graphiques sur une même figure : Subplots
          \end{itemize}
        \end{block}
      \end{column}
  
      \begin{column}{.48\textwidth}
        \begin{center}
          \includegraphics[width=\textwidth]{img/Figure_Axes}
        \end{center}
      \end{column}
    \end{columns}

    \begin{block}{Convention d'import}
\begin{lstlisting}[language=Python, numbers=none]
import matplotlib.pyplot as plt
\end{lstlisting}
    \end{block}

  \end{frame}


  \begin{frame}[fragile]{\includegraphics[width=.2\textwidth]{img/logo_matplotlib.png} Figures, Axes}
  
      \begin{block}{Figure}
          \begin{itemize}
            \item La ``fenêtre'' contenant tous les objets composant le ou les graphiques
            \item Un objet Figure peut contenir : titre, sous-titre, légende, etc.
            \item Une figure est créée comme suit :
          \end{itemize}
      \end{block}
\begin{lstlisting}[language=Python, numbers=none]
fig = plt.figure()
\end{lstlisting}
  
    \begin{block}{Axes}
        \begin{itemize}
          \item Dans une figure, on ajoute un ou plusieurs objets \textcolor{orange}{Axes}
          \item Un tel objet est une \textcolor{orange}{r\'egion} de l'image qui recevra les \textcolor{orange}{donn\'es à visualiser}
          \item Une figure peut contenir plusieurs axes
          \item Un axe contient deux ou trois objets \textcolor{orange}{Axis}
        \end{itemize}
    \end{block}
  \end{frame}


  
  
  \begin{frame}[fragile]{\includegraphics[width=.2\textwidth]{img/logo_matplotlib} Figures and Axes}
\begin{lstlisting}[language=Python, morekeywords={as}, numbers=none]
import matplotlib.pyplot as plt

fig = plt.figure() # Create the Figure object

# Add an Axe to this figure
ax = fig.add_axes([0.5,0.5,0.8,0.8]) # dimensions [left, bottom, width, height] of the new axes

# Plot the distribution of the wind variable in the axe
g1 = ax.hist(x=seattle_weather['wind'])

# Add a title and labels for the x-axis and the y-axis
ax.set_title("Histogram with default parameters")
ax.set_xlabel("Wind")
ax.set_ylabel("Count")

plt.show() # Show the plot
\end{lstlisting}
  \end{frame}
  
  \begin{frame}[fragile]{\includegraphics[width=.2\textwidth]{img/logo_matplotlib} Figures and Axes}
\begin{lstlisting}[language=Python, morekeywords={as}, numbers=none]
# Short version: figure and axes objects are automatically created

import matplotlib.pyplot as plt

# Plot the distribution of the wind using a histogram
plt.hist(x=seattle_weather["wind"])

# Add a title
plt.title("Histogram with default parameters")

# Add labels for the x-axis and the y-axis
plt.xlabel("Wind")
plt.ylabel("Count")

plt.show() # Show the plot
\end{lstlisting}
  \end{frame}
  
  \begin{frame}{\includegraphics[width=.2\textwidth]{img/logo_matplotlib} Subplots}
    \begin{block}{Qu'est-ce qu'un subplot?}
      \begin{itemize}
        \item La méthode \texttt{matplotlib.pyplot.subplots} crée plusieurs graphiques \textcolor{orange}{sur une même figure}.
        \item Elle retourne un tuple (fig, ax) en fonction du nombre de lignes et de colonnes souhaitées.
      \end{itemize}
    \end{block}
    Plus de détails
    \begin{itemize}
      \item \href{https://www.educative.io/edpresso/what-is-a-subplots-in-matplotlib}{educative.io/edpresso/what-is-a-subplots-in-matplotlib}
      \item \href{https://matplotlib.org/stable/gallery/subplots_axes_and_figures/subplots_demo.html}{matplotlib.org/stable/gallery/subplots\_axes\_and\_figures}
    \end{itemize}
  \end{frame}
  
  
  \begin{frame}[fragile]{\includegraphics[width=.2\textwidth]{img/logo_matplotlib} Subplots}
  
    \begin{block}{Usage 1}
\begin{lstlisting}[language=Python, morekeywords={as}, numbers=none]
import matplotlib.pyplot as plt

# Create a subplot with 1 row and 2 columns
fig, (ax1, ax2) = plt.subplots(1, 2)

# Hist on the 1st column
ax1.hist(x)

# Scatter on the 2nd column
ax2.scatter(x, y)

# Show the plot
plt.show()
\end{lstlisting}
    \end{block}
  
  \end{frame}
  
  \begin{frame}[fragile]{\includegraphics[width=.2\textwidth]{img/logo_matplotlib} Subplots}
  
    \begin{block}{Usage 2}
\begin{lstlisting}[language=Python, morekeywords={as}, numbers=none]
import matplotlib.pyplot as plt

# Create the figure
fig = plt.figure()

# Create an histogram on the 1st subplot
plt.subplot(121) # or plt.subplot(1, 2, 1)
plt.hist(x)

# Create scatter plot on the 2nd subplot
plt.subplot(122) # or plt.subplot(1, 2, 2)
plt.scatter(x, y)

# Show the plot
plt.show()
\end{lstlisting}
    \end{block}
  
  \end{frame}



  \begin{frame}{Seaborn: high-level interface of matplotlib}

    \begin{block}{Overview of seaborn}
      \medskip
      Seaborn is a \textcolor{blue}{complementary} library for Matplotlib, which provides a high-level \textcolor{blue}{interface} to draw \textcolor{blue}{statistical} graphics.
  
      It uses Pandas data frames
    \end{block}
  
    \begin{block}{Useful links}
      \medskip
      \begin{itemize}
        \item Gallery: \href{https://seaborn.pydata.org/examples/index.html}{seaborn.pydata.org/examples}
        \item Tutorial: \href{https://seaborn.pydata.org/tutorial.html}{seaborn.pydata.org/tutorial}
      \end{itemize}
      
    \end{block}
  \end{frame}
  
  
  
  \begin{frame}{Seaborn: high-level interface of matplotlib}
  
    \begin{columns}[T]
      \begin{column}{.25\textwidth}
       \medskip
       \centering
       \textsc{Figure-level}
       
       vs
       
       \textsc{Axes-level}
      \end{column}
  
      \begin{column}{.25\textwidth}
        \medskip
        \centering
        \textsc{Relational}
  
        \textsc{Distributional}
  
        \textsc{Categorical}
      \end{column}
  
    \end{columns}
  
    \centering
    \includegraphics[width=.7\textwidth]{/Users/baptistegregorutti/Documents/Repositories/gitlab/data-visualization-slides/dataviz_python/img/seaborn-overview.png}
  \end{frame}
  
  
  
  
  \begin{frame}[fragile,t]{Figure-level vs Axes-level functions}
  
    \begin{block}{Axes-level}
      \medskip
      Axes-level functions plots data onto a single matplotlib Axes object ax
    \end{block}
  
\begin{lstlisting}[language=python, numbers=none]
# creates a chart in an existing axes object
sns.histplot(data=df, x="temperature", ax=ax)
\end{lstlisting}
  
    \onslide<2>{
      \begin{block}{Figure-level}
        \medskip
        Figure-level functions interface with matplotlib through a seaborn object that manages the figure.
        
        For each task, the figure-level function is an interface to various axes-level functions
        
        % For example, the figure-level function `displot` is an interface to the axes-level functions `histplot`, `kdeplot`, etc. see the figure below.
      \end{block}
    }
    \begin{onlyenv}<2>
\begin{lstlisting}[language=python, numbers=none]
sns.displot(data=df, x="temperature", kind="hist")
\end{lstlisting}    
    \end{onlyenv}
  \end{frame}
  
  
  
  
  \begin{frame}{Figure-level vs Axes-level functions}
  
    \begin{columns}[T]
      \begin{column}{.25\textwidth}
       \medskip
       \centering
       \textsc{Figure-level}
       
       vs
       
       \textsc{Axes-level}
      \end{column}
  
      \begin{column}{.25\textwidth}
  
      \end{column}
  
    \end{columns}
  
    \centering
    \includegraphics[width=.7\textwidth]{/Users/baptistegregorutti/Documents/Repositories/gitlab/data-visualization-slides/dataviz_python/img/seaborn-overview.png}
  \end{frame}
  
  
  \begin{frame}[fragile]{Same usage / Different tasks}
  
    \begin{block}{The functions have (almost) the same signature}
      \medskip
      \begin{itemize}
        \item data: a pandas DataFrame
        \item x, y: columns names in the data frame
        \item hue: name of the column used to add colors
        \item ax: maplotlib Axes object in which to draw the plot
      \end{itemize}
      
    \end{block}
  
\begin{lstlisting}[language=python, numbers=none]
# Scatter plot
sns.scatterplot(data=df, x="time", y="temperature", ax=ax)

# Line plot
sns.lineplot(data=df, x="time", y="temperature", ax=ax)

# Line plot, Figure-level function
sns.relplot(data=df, x="time", y="temperature", kind="line")
\end{lstlisting}
  
  \end{frame}
  
  \begin{frame}[fragile]{Statistical graphics}
  
    Seaborn integrates statistics into the graphics making easier the analysis of the data.
  
    \begin{itemize}
      \item Automatic computation of statistics: mean, confidence interval, error bars, etc.
      \item Automatic linear regression fitting
    \end{itemize}
    
    \begin{center}
      \includegraphics[width=.4\textwidth]{/Users/baptistegregorutti/Documents/Repositories/gitlab/data-visualization-slides/dataviz_python/img/cat-error-bar.png}
      \includegraphics[width=.4\textwidth]{/Users/baptistegregorutti/Documents/Repositories/gitlab/data-visualization-slides/dataviz_python/img/reg-fit.png}
    \end{center}
  
  \end{frame}
  
  \begin{frame}[fragile]{Figure aesthetics}
    The aesthetics can be modified, helping the communication of insights
  
    The function \texttt{set\_theme} modifes the background, the color palette, the font, etc.

\begin{lstlisting}[language=python, numbers=none]
# Default theme
sns.set_theme()

# Dark background with a grid
sns.set_theme(style="darkgrid")

# White background with a grid
sns.set_theme(style="whitegrid", palette="pastel")
\end{lstlisting}
    
    \begin{center}
      \includegraphics[width=.3\textwidth]{/Users/baptistegregorutti/Documents/Repositories/gitlab/data-visualization-slides/dataviz_python/img/set-theme-1.png}
      \includegraphics[width=.3\textwidth]{/Users/baptistegregorutti/Documents/Repositories/gitlab/data-visualization-slides/dataviz_python/img/set-theme-2.png}
      \includegraphics[width=.3\textwidth]{/Users/baptistegregorutti/Documents/Repositories/gitlab/data-visualization-slides/dataviz_python/img/set-theme-3.png}
    \end{center}
  
  \end{frame}


  
  
  
  \begin{frame}[fragile]{Some examples}
  
    \begin{block}{Plotting distributions}
\begin{lstlisting}[language=python, numbers=none]
# Axes-level histogram
sns.histplot(x="value", hue="variable", data=df)

# Figure-level histogram
sns.displot(x="value", hue="variable", data=df)
\end{lstlisting}
    \end{block}
  
    \begin{center}
      \includegraphics[width=.4\textwidth]{img/hist_axe.png}
      \includegraphics[width=.4\textwidth]{img/hist_figure.png}
    \end{center}
  
  \end{frame}
  
\begin{frame}[fragile]{Some examples}
  
  \begin{block}{Plotting distributions}
\begin{lstlisting}[language=python, numbers=none]
# Axes-level histogram
sns.kdeplot(x="value", hue="variable", data=df)

# Figure-level histogram
sns.displot(x="value", hue="variable", data=df, kind="kde")
\end{lstlisting}
  \end{block}
  
  \begin{center}
    \includegraphics[width=.4\textwidth]{img/kde_axe.png}
    \includegraphics[width=.4\textwidth]{img/kde_figure.png}
  \end{center}

\end{frame}

  
  
  \begin{frame}[fragile]{Some examples}
  
    \begin{block}{}
\begin{lstlisting}[language=python, numbers=none]
# Pairplot / Matrix of scatterplots
sns.pairplot(x)

# Joint plot
sns.jointplot(x)
\end{lstlisting}
    \end{block}
  
    \begin{center}
      \includegraphics[width=.4\textwidth]{img/scatterplot_matrix}
      \includegraphics[width=.4\textwidth]{img/joint}
    \end{center}
  
\end{frame}


    
\begin{frame}{Seaborn: high-level interface of matplotlib}
  \begin{center}
    \includegraphics[width=\textwidth]{/Users/baptistegregorutti/Documents/Repositories/gitlab/data-visualization-slides/dataviz_python/img/seaborn-cheat.png}
  \end{center}
  \tiny{Source: interactivechaos.com}
\end{frame}




\end{document}