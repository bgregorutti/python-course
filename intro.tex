\begin{frame}{}
    \centering
    \Large
    \textbf{Introduction}
\end{frame}

\begin{frame}{Introduction}
  \begin{block}{Historique}
    \medskip
    Le langage Python a été créé par Guido van Rossum en 1989 et rendu public en 1991. Le nom fait référence aux {\it Monty Python}.

    G. van Rossum a été jusqu'à 2018 ``Benevolent Dictator for Life''.
  \end{block}

  \bigskip
  \centering
  \url{www.python.org}
\end{frame}

\begin{frame}{Un langage interprété...}

  \begin{columns}[T]
    \begin{column}{.6\textwidth}
      \begin{block}{}
        \begin{itemize}
          \item<+-> Simplificité d'écriture et flexibilité
          \item<+-> Exécution par un interpréteur ligne par ligne
          \item<+-> Exécution intéractive ou par script
          \item<+-> Multiplateforme et open source
          \item<+-> Correction de bugs \textbf{relativement simple}
        \end{itemize}
      \end{block}
    \end{column}%
    \hfill
    \begin{column}{.4\textwidth}
      \includegraphics[width=0.85\textwidth]{img/abstraction}
    \end{column}%
  \end{columns}

\end{frame}



\begin{frame}{... mais relativement peu performant}

  \begin{columns}[T]
    \begin{column}{.6\textwidth}
      \begin{block}{}
        \begin{itemize}
          \item<+-> Interprétation vs compilation
          \item<+-> Gestion de données et bibliothèques de haut niveau
          \item<+-> Typage dynamique
          \item<+-> Gestion automatique de mémoire
          \item<+-> Traitement des boucles
        \end{itemize}
      \end{block}
    \end{column}%
    \hfill
    \begin{column}{.4\textwidth}
      \includegraphics[width=0.85\textwidth]{img/abstraction}
    \end{column}%
  \end{columns}

\end{frame}







\begin{frame}{Une histoire de versions}
  \centering
  \includegraphics[width=\textwidth]{img/python_versions.png}
\end{frame}



\section{Installation et revue des outils}




\begin{frame}{Pré-requis pour le cours}

  \begin{block}{}
    \medskip
    \begin{itemize}
      \item Miniconda / Anaconda, python $>=$ 3.8
      \item Un terminal
      \begin{itemize}
        \item Linux / MaxOS : le terminal par défaut
        \item Windows : privilégier Powershell, intégré à Anaconda
      \end{itemize}
      \item jupyter-lab ou jupyter-notebook
      \item Un IDE (Atom, VScode, etc.)
    \end{itemize}
  \end{block}
 
  \end{frame}



\begin{frame}{Commandes Linux à connaitre}
  \begin{center}
    \includegraphics[width=\textwidth]{img/List-of-basic-Linux-commands.png}
  \end{center}
\end{frame}



\begin{frame}[fragile]{Installation de Python}
  Il existe plusieurs façons d'installer python (et ses outils).

\begin{overprint}
  \onslide<1>
    \begin{block}{1/ Paquets Linux}
    \medskip
\begin{lstlisting}[language=bash, morekeywords=\$, numbers=none]
$ sudo apt-get install python3 python3-pip
\end{lstlisting}
    \begin{itemize}
      \item Python3 est en général déjà installé
      \item Sous Linux, \textbf{Python2 est systématiquement installé}
    \end{itemize}
    \end{block}

  \onslide<2>
    \begin{block}{2/ Anaconda}
    \medskip
      \begin{itemize}
        \item Installation complète de l'environnement
        \item Un très grand nombre de libraries pré-installées
        \item Multiplateforme : Windows, Linux, MacOS
      \end{itemize}
    \end{block}
    Voir : \url{https://docs.anaconda.com/anaconda/install}

  \onslide<3>
    \begin{block}{3/ Miniconda}
      \medskip
      \begin{itemize}
        \item Installation minimale de l'environnement
        \item Aucune librairie installée
        \item Multiplateforme : Windows, Linux, MacOS
      \end{itemize}
    \end{block}
    Voir : \url{https://docs.anaconda.com/anaconda/install}
\end{overprint}
\end{frame}



% \begin{frame}[fragile]{Installation de Python}
%   Une fois Python installé, ouvrir un terminal et écrire
% \begin{lstlisting}[language=bash, morekeywords=\$, numbers=none]
% $ python --version
% $ python3 --version
% \end{lstlisting}
% \end{frame}





\begin{frame}{Installation de Python via Anaconda}

  \begin{center}
   \includegraphics[width=\textwidth]{img/anaconda.png}
  \end{center}
   
 \end{frame}
 
 
 
%  \begin{frame}{Installation de Python via Anaconda}
%    \begin{block}{Anaconda}
%      \medskip
%        \begin{itemize}
%          \item Installation complète de l'environnement
%          \item Un très grand nombre de libraries pré-installées
%          \item Multiplateforme : Windows, Linux, MacOS
%        \end{itemize}
%    \end{block}

%  \end{frame}
 
 
 \begin{frame}{Installation de Python via Anaconda}
   \begin{center}
     \includegraphics[width=\textwidth]{img/anaconda_navigator.png}
   \end{center}
 \end{frame}
 
 
 \begin{frame}{Jupyter}
   \begin{center}
     \includegraphics[width=\textwidth]{img/jupyter-nb.png}
   \end{center}
 \end{frame}
 
 \begin{frame}{Jupyter}
   \begin{center}
     \includegraphics[width=\textwidth]{img/jupyter-lab-1.png}
   \end{center}
 \end{frame}
 
 \begin{frame}{Jupyter}
   \begin{center}
     \includegraphics[width=\textwidth]{img/jupyter-lab-2.png}
   \end{center}
 \end{frame}
 
 \begin{frame}{Jupyter}
   \begin{center}
     \includegraphics[width=\textwidth]{img/jupyter-lab-3.png}
   \end{center}
 \end{frame}


\begin{frame}[fragile]{Exécution d'un programme python : alternatives}

  \begin{block}{Cas 1 : via un shell python ou ipython}
      \medskip
      Exécution à la volée d'instructions python. Ouvrir un shell
\begin{lstlisting}[language=bash, morekeywords=\$, numbers=none]
$ python
\end{lstlisting}
      Ecrire des instructions
\begin{lstlisting}[language=python, numbers=none]
>>> print("Je suis un programme python")
\end{lstlisting}
  \end{block}

  \begin{center}
    \includegraphics[width=0.7\textwidth]{img/print_shell.png}
  \end{center}

\end{frame}
  

\begin{frame}[fragile]{Exécution d'un programme python : alternatives}

  \begin{block}{Cas 2 : via un terminal}
    \medskip
    Dans un éditeur de texte, créer un fichier \texttt{program.py} et écrire
\begin{lstlisting}[language=python, numbers=none]
print("Je suis un programme python")
\end{lstlisting}

    Exécuter la ligne de commande
\begin{lstlisting}[language=bash, morekeywords=$, numbers=none]
$ python program.py
\end{lstlisting}
  \end{block}

  \begin{center}
    \includegraphics[width=0.7\textwidth]{img/dummy_program}
  \end{center}
\end{frame}


\begin{frame}{Revue des outils}

  \onslide<1->{
    \begin{block}{Outils de base :}
      \medskip
      \begin{itemize}
        \item python : exécuter du code à la volée
        \item pip : installer des packages
        \item pytest : tests unitaires
        \item pylint : vérification de la qualité du code source
      \end{itemize}
    \end{block}
  }
  \onslide<2>{
    \begin{block}{Intégré au projet Anaconda :}
      \medskip
      \begin{itemize}
        \item ipython : version interactive du shell python
        \item conda : installer des packages, gérer des environnements d'exécutions (voir \href{https://conda.io/projects/conda/en/latest/user-guide/tasks/manage-environments.html}{\color{blue}{\underline{ici}}}), etc.
        \item Jupyter : environnement de développement intéractif et fléxible
      \end{itemize}
    \end{block}
  }
  \end{frame}
  

\begin{frame}[fragile]{Pour aller plus loin : environnements conda}
  % Pour gérer des environnements virtuels, avec des installations spécifiques
  \begin{center}
    \includegraphics[width=.7\textwidth]{img/conda-envs.png}
  \end{center}
% \begin{lstlisting}[language=bash, morekeywords=\$, numbers=none]
% $ conda create --name myenv python=3.9
% $ conda activate myenv
% $ conda install pip
% $ pip install numpy
% $ conda deactivate
% \end{lstlisting}
Voir \href{https://conda.io/projects/conda/en/latest/user-guide/tasks/manage-environments.html}{\color{blue}{\underline{ici}}}
\end{frame}