% !TeX root = part_1.tex

\begin{frame}{}
    \centering
    \Large
    \textbf{Introduction}
\end{frame}

\begin{frame}{Introduction}
  \begin{block}{Historique}
    \medskip
    Le langage Python a été créé par Guido van Rossum en 1989 et rendu public en 1991. Le nom fait référence aux {\it Monty Python}.

    G. van Rossum a été jusqu'à 2018 ``Benevolent Dictator for Life''.
  \end{block}

  \bigskip
  \centering
  \url{www.python.org}
\end{frame}

\begin{frame}{Un langage interprété...}

  \begin{columns}[T]
    \begin{column}{.6\textwidth}
      \begin{block}{}
        \begin{itemize}
          \item<+-> Simplificité d'écriture et flexibilité
          \item<+-> Exécution par un interpréteur ligne par ligne
          \item<+-> Exécution intéractive ou par script
          \item<+-> Multiplateforme et open source
          \item<+-> Correction de bugs \textbf{relativement simple}
        \end{itemize}
      \end{block}
    \end{column}%
    \hfill
    \begin{column}{.4\textwidth}
      \includegraphics[width=0.85\textwidth]{img/abstraction}
    \end{column}%
  \end{columns}

\end{frame}



\begin{frame}{... mais relativement peu performant}

  \begin{columns}[T]
    \begin{column}{.6\textwidth}
      \begin{block}{}
        \begin{itemize}
          \item<+-> Interprétation vs compilation
          \item<+-> Gestion de données et bibliothèques de haut niveau
          \item<+-> Typage dynamique
          \item<+-> Gestion automatique de mémoire
          \item<+-> Traitement des boucles
        \end{itemize}
      \end{block}
    \end{column}%
    \hfill
    \begin{column}{.4\textwidth}
      \includegraphics[width=0.85\textwidth]{img/abstraction}
    \end{column}%
  \end{columns}

\end{frame}



  \begin{frame}[fragile]{Interprété vs compilé}

    Différences entre langage interprété et compilé :
    \begin{itemize}
      \item<+-> Langage \textcolor{blue}{compilé} : traduction complète du code source en code machine avant exécution
      \item<+-> Langage \textcolor{red}{interprété} : traduction ligne par ligne du code source en code machine pendant l'exécution
    \end{itemize}

    \bigskip

    \begin{overprint}
      \onslide<1>
      Exemple de workflow en C++ :
\begin{lstlisting}[language=bash, morekeywords=\$, numbers=none]
$ g++ -c program.cpp -o program.o # Compilation
$ g++ program.o -o program        # Edition de liens
$ ./program                       # Execution
\end{lstlisting}

      \onslide<2>
      Exemple de workflow en Python :
\begin{lstlisting}[language=bash, morekeywords=\$, numbers=none]
$ python program.py # Interpretation et execution
\end{lstlisting}
    \end{overprint}

  \end{frame}

  \begin{frame}{Une histoire de versions}
    \centering
    \includegraphics[width=\textwidth]{img/python_versions.png}
  \end{frame}



  \section{Installation et revue des outils}

  \begin{frame}{Installation de Python}

    Plusieurs alternatives pour installer Python :
    \begin{itemize}
      \item Installation manuelle depuis \href{www.python.org}{www.python.org} ou via un gestionnaire de paquets (Linux, MacOS)
      \item Installation via une distribution scientifique (Anaconda, Miniconda)
    \end{itemize}

    \bigskip
    Installation recommandée via Anaconda, qui inclut un grand nombre de packages scientifiques et des outils de développement.
  \end{frame}

\begin{frame}{Installation de Python via Anaconda}

  \begin{center}
   \includegraphics[width=\textwidth]{img/anaconda.png}
  \end{center}
   
 \end{frame}
 
 
 
%  \begin{frame}{Installation de Python via Anaconda}
%    \begin{block}{Anaconda}
%      \medskip
%        \begin{itemize}
%          \item Installation complète de l'environnement
%          \item Un très grand nombre de libraries pré-installées
%          \item Multiplateforme : Windows, Linux, MacOS
%        \end{itemize}
%    \end{block}

%  \end{frame}
 
 
 \begin{frame}{Installation de Python via Anaconda}
   \begin{center}
     \includegraphics[width=\textwidth]{img/anaconda_navigator.png}
   \end{center}
 \end{frame}
 
 
 \begin{frame}{Jupyter}
   \begin{center}
     \includegraphics[width=\textwidth]{img/jupyter-nb.png}
   \end{center}
 \end{frame}
 
 \begin{frame}{Jupyter}
   \begin{center}
     \includegraphics[width=\textwidth]{img/jupyter-lab-1.png}
   \end{center}
 \end{frame}
 
 \begin{frame}{Jupyter}
   \begin{center}
     \includegraphics[width=\textwidth]{img/jupyter-lab-2.png}
   \end{center}
 \end{frame}
 
 \begin{frame}{Jupyter}
   \begin{center}
     \includegraphics[width=\textwidth]{img/jupyter-lab-3.png}
   \end{center}
 \end{frame}

 
 \begin{frame}{VSCode}
   \begin{center}
     \includegraphics[width=\textwidth]{img/vscode.png}
   \end{center}
 \end{frame}


\begin{frame}{Revue des outils}

  \onslide<1->{
    \begin{block}{Outils de base :}
      \medskip
      \begin{itemize}
        \item python : exécuter du code à la volée
        \item pip : installer des packages
        \item pytest : tests unitaires
        \item pylint : vérification de la qualité du code source
      \end{itemize}
    \end{block}
  }
  \onslide<2>{
    \begin{block}{Intégré au projet Anaconda :}
      \medskip
      \begin{itemize}
        \item ipython : version interactive du shell python
        \item conda : installer des packages, gérer des environnements d'exécutions (voir \href{https://conda.io/projects/conda/en/latest/user-guide/tasks/manage-environments.html}{\color{blue}{\underline{ici}}}), etc.
        \item Jupyter : environnement de développement intéractif et fléxible
      \end{itemize}
    \end{block}
  }
  \end{frame}
  

\begin{frame}[fragile]{Pour aller plus loin : environnements conda}
  % Pour gérer des environnements virtuels, avec des installations spécifiques
  \begin{center}
    \includegraphics[width=.7\textwidth]{img/conda-envs.png}
  \end{center}
% \begin{lstlisting}[language=bash, morekeywords=\$, numbers=none]
% $ conda create --name myenv python=3.9
% $ conda activate myenv
% $ conda install pip
% $ pip install numpy
% $ conda deactivate
% \end{lstlisting}
Voir \href{https://conda.io/projects/conda/en/latest/user-guide/tasks/manage-environments.html}{\color{blue}{\underline{ici}}}
\end{frame}