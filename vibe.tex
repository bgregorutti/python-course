\section{Vibe coding}

\begin{frame}[fragile]{Qu'est-ce que ``vibe coding'' ?}
    \begin{block}{Définition}
        \medskip
        Utiliser des systèmes d'IA pour aider à écrire, expliquer, tester et refactorer du code afin d'améliorer la qualité et l'agrément de lecture du projet.
    \end{block}
   \bigskip 
   \begin{center}
        \includegraphics[width=0.1\textwidth]{img/openai-logo.png}
        \includegraphics[width=0.1\textwidth]{img/claude-logo.png}
        \includegraphics[width=0.1\textwidth]{img/copilot-logo.png}
        \includegraphics[width=0.1\textwidth]{img/lmstudio-logo.jpg}
   \end{center}
\end{frame}

\begin{frame}[fragile]{Qu'est-ce que ``vibe coding'' ?}
    \begin{center}
        \includegraphics[width=0.6\textwidth]{img/vibe-coding.png}
    \end{center}
\end{frame}


\begin{frame}[fragile]{Qu'est-ce que ``vibe coding'' ?}

    \onslide<1->{
        \begin{block}{Avantages}
            \medskip
            \begin{itemize}
                \item Accélère le \textcolor{blue}{prototypage} (tests, docstrings, templates)
                \item Aide à découvrir des bonnes pratiques
                \item Facilite la \textcolor{blue}{revue} rapide, la génération d'exemples et la documentation
                \item Permet \textcolor{blue}{d'automatiser des tâches répétitives} et proposer des refactorings
            \end{itemize}
        \end{block}
    }

    \onslide<2->{
        \begin{block}{Inconvénients / risques}
            \medskip
            \begin{itemize}
                \item Résultats incorrects ou non sécurisés — \textcolor{red}{toujours valider}
                \item Risques liées \textcolor{red}{confidentialité} et à la \textcolor{red}{dépense}.
                \item \textcolor{red}{Dépendance excessive} pouvant réduire l'apprentissage et la compréhension du code.
                \item Variabilité et difficulté de reproductibilité selon le modèle/version du service.
            \end{itemize}
        \end{block}
    }
\end{frame}

\begin{frame}{Bonnes pratiques pour le vibe coding}
    
    \begin{block}{Bonnes pratiques}
        \medskip
        \begin{itemize}
            \item<+-> \textbf{Valider} toutes les suggestions par des revues et des tests automatisés
            \item<+-> Ne jamais envoyer de \textbf{données sensibles} dans les prompts ; préférer des \textbf{modèles locaux} pour du code privé
            \item<+-> Utiliser l'IA comme \textbf{assistant} (productivité + suggestions), pas comme remplacement du jugement humain
            \item<+-> Intégrer IA dans un \textbf{workflow} avec linter, tests, CI et revue de code.
        \end{itemize}
    \end{block}
\end{frame}

% \begin{frame}{Quid de l'empreinte carbonne de l'IA ?}

%     \begin{block}{L'IA n'est pas neutre !}
%         \medskip
%         \begin{itemize}
%             \item Empreinte générée par : entraînements (GPU/TPU intensifs), inférence à grande échelle, datacenters et fabrication du matériel (cycle de vie)
%             \item Facteurs clés : consommation énergétique (kWh), source d'énergie (fossile vs renouvelable), durée/fréquence des entraînements, efficience des modèles.
%             \item Conséquences : émissions de CO2e, consommation d'eau pour refroidissement, impact lié à la fabrication et fin de vie du matériel.
%             \item Mesures d'atténuation :
%                 \begin{itemize}
%                     \item Optimiser les entraînements (profiling, mixed precision, batch sizing).
%                     \item Réduire la taille des modèles (distillation, pruning, quantization) et réutiliser les modèles existants.
%                     \item Planifier calculs et inférences dans des régions/centres alimentés par des énergies renouvelables.
%                     \item Mettre en place métriques et rapports (kWh, kg CO2e) pour suivre l'impact.
%                     \item Améliorer l'efficacité logicielle et matérielle, et limiter les runs expérimentaux inutiles.
%                 \end{itemize}
%             \item Bonnes pratiques : peser bénéfices vs coût carbone, intégrer l'empreinte dans les revues/CI, privilégier transparence et modèles responsables.
%         \end{itemize}
%     \end{block}


    
%     \vfill
%     \small Ref : \url{https://bonpote.com/intelligence-artificielle-le-vrai-cout-environnemental-de-la-course-a-lia/}

% \end{frame}



